%% =============================================================================
%% configuracion.tex
%% Archivo de configuración global del usuario
%%
%% Proyecto: Plantilla TFG/TFM EPS Universidad de Alicante
%% Autor:    José Manuel Requena Plens
%% Versión:  2.1.0 (2026/02/05)
%% =============================================================================
%
% Este archivo contiene toda la información personal y del trabajo.
% Modifica los valores según corresponda a tu situación.
%

\EPSsetup{
  %% =====================================================================
  %% INFORMACIÓN DEL TRABAJO
  %% =====================================================================
  
  % Título del trabajo (obligatorio)
  titulo = {Título del Trabajo Fin de Grado/Máster},
  
  % Subtítulo (opcional, se muestra en portada B/N)
  subtitulo = {Subtítulo del proyecto},
  
  %% =====================================================================
  %% INFORMACIÓN DEL AUTOR
  %% =====================================================================
  
  % Nombre completo del autor (obligatorio)
  autor = {Nombre Apellido1 Apellido2},
  
  % Género para etiquetas: m (masculino), f (femenino), n (neutro)
  genero = m,
  
  % Email institucional
  email = nombre@alu.ua.es,
  
  %% =====================================================================
  %% INFORMACIÓN DEL TUTOR/ES
  %% =====================================================================
  
  % Tutor principal (obligatorio)
  tutor = {Dr./Dra. Nombre Apellido1 Apellido2},
  tutor-genero = m,  % m = masculino, f = femenino, n = neutro
  tutor-departamento = {Departamento de ejemplo},
  
  % Co-tutor (opcional - descomentar si hay cotutor)
  % cotutor = {Dr./Dra. Nombre Apellido1 Apellido2},
  % cotutor-genero = m,  % m = masculino, f = femenino, n = neutro
  % cotutor-departamento = {Departamento del cotutor},
  
  %% =====================================================================
  %% TITULACIÓN
  %% =====================================================================
  % 
  % Grados disponibles:
  %   teleco            - Ingeniería en Sonido e Imagen en Telecomunicación
  %   civil             - Ingeniería Civil
  %   quimica           - Ingeniería Química
  %   informatica       - Ingeniería Informática
  %   multimedia        - Ingeniería Multimedia
  %   arquitectura-tecnica - Arquitectura Técnica
  %   arquitectura      - Arquitectura
  %   robotica          - Ingeniería Robótica
  %
  % Másteres disponibles:
  %   master-teleco     - Ingeniería de Telecomunicación
  %   master-caminos    - Caminos, Canales y Puertos
  %   master-edificacion - Gestión de la Edificación
  %   master-web        - Desarrollo de Aplicaciones y Servicios Web
  %   master-materiales - Materiales, Agua y Terreno
  %   master-informatica - Ingeniería Informática
  %   master-robotica   - Automática y Robótica
  %   master-prevencion - Prevención de Riesgos Laborales
  %   master-agua       - Gestión Sostenible y Tecnologías del Agua
  %   master-moviles    - Software para Dispositivos Móviles
  %   master-quimica    - Ingeniería Química
  %   master-ciberseguridad - Ciberseguridad
  %   master-geologica  - Ingeniería Geológica
  %
  titulacion = teleco,
  
  %% =====================================================================
  %% IDIOMA DEL DOCUMENTO
  %% =====================================================================
  %
  % Idioma principal del trabajo. Afecta a:
  % - Títulos de secciones automáticas (Tabla, Figura, Bibliografía, etc.)
  % - Formato de citas y bibliografía
  % - Metadatos del PDF
  %
  % Valores disponibles:
  %   espanol    - Español (predeterminado)
  %   valenciano - Valenciano/Catalán
  %   ingles     - Inglés
  %
  % IMPORTANTE: Si cambias el idioma, actualiza también el archivo
  % 'cls/eps-metadata.tex' para que coincida (ver comentarios allí).
  %
  idioma = espanol,
  
  %% =====================================================================
  %% INFORMACIÓN INSTITUCIONAL (normalmente no hay que cambiar)
  %% =====================================================================
  
  facultad = {Escuela Politécnica Superior},
  universidad = {Universidad de Alicante},
  ubicacion = {Alicante},
  
  %% =====================================================================
  %% OPCIONES ADICIONALES
  %% =====================================================================
  
  % Fecha personalizada (si se omite, usa la fecha actual)
  % fecha = {Junio 2026},
  
  % Optimizar figuras TikZ (guardar en caché)
  optimizar-tikz = true,
  
  % Modo borrador (muestra notas TODO)
  borrador = true,
}
