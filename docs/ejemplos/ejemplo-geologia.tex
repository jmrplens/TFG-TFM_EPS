% =============================================================================
% Ejemplo autocontenido de componentes de Geología
% Compilar con: lualatex -shell-escape ejemplo-geologia.tex
% =============================================================================
\documentclass[12pt,a4paper]{article}

% Configuración básica
\usepackage[spanish]{babel}
\usepackage{fontspec}
\usepackage{xcolor}
\usepackage{tcolorbox}
\tcbuselibrary{skins,breakable}
\usepackage{booktabs}
\usepackage{siunitx}
\usepackage{tikz}
\usetikzlibrary{positioning,shapes,patterns,calc,decorations.pathmorphing}

% Definición de colores de la paleta
\definecolor{eps-primary}{HTML}{2563EB}
\definecolor{eps-secondary}{HTML}{7C3AED}
\definecolor{eps-success}{HTML}{059669}
\definecolor{eps-warning}{HTML}{D97706}
\definecolor{eps-danger}{HTML}{DC2626}
\definecolor{eps-info}{HTML}{0284C7}
\definecolor{eps-dark}{HTML}{1F2937}
\definecolor{eps-gray}{HTML}{6B7280}
\definecolor{eps-light}{HTML}{F3F4F6}

% Colores geológicos
\definecolor{igneous-color}{HTML}{DC2626}
\definecolor{sedimentary-color}{HTML}{D97706}
\definecolor{metamorphic-color}{HTML}{7C3AED}
\definecolor{soil-color}{HTML}{92400E}
\definecolor{water-color}{HTML}{0284C7}

% --- COMPONENTES DE GEOLOGÍA ---

% Ficha de roca
\newtcolorbox{rockcard}[2][]{%
  colback=white,
  colframe=#2,
  title={\textbf{Ficha de roca}},
  fonttitle=\small\sffamily,
  boxrule=0.5pt,
  #1
}

% Badge de tipo de roca
\newcommand{\rockbadge}[2]{%
  \tikz[baseline=(text.base)]{%
    \node[fill=#2!15,draw=#2,rounded corners=2pt,inner sep=3pt,font=\sffamily\footnotesize\bfseries,text=#2] (text) {#1};
  }%
}

\newcommand{\igneousbadge}{\rockbadge{Ígnea}{igneous-color}}
\newcommand{\sedimentarybadge}{\rockbadge{Sedimentaria}{sedimentary-color}}
\newcommand{\metamorphicbadge}{\rockbadge{Metamórfica}{metamorphic-color}}

% Propiedades geotécnicas
\newenvironment{geoprops}[1][Propiedades geotécnicas]{%
  \begin{tcolorbox}[
    colback=eps-light,
    colframe=eps-dark,
    title={\textbf{#1}},
    fonttitle=\small\sffamily,
    boxrule=0.5pt
  ]
  \small
  \begin{tabular}{@{}ll@{}}
}{%
  \end{tabular}
  \end{tcolorbox}
}

% Columna estratigráfica
\newenvironment{stratigraphiccolumn}[1][Columna estratigráfica]{%
  \begin{tcolorbox}[
    colback=white,
    colframe=soil-color,
    title={\textbf{#1}},
    fonttitle=\small\sffamily,
    boxrule=0.5pt
  ]
  \centering
}{%
  \end{tcolorbox}
}

% Capa estratigráfica
\newcommand{\stratlayer}[4]{% {color}{espesor}{nombre}{edad}
  \tikz{%
    \fill[#1!30,draw=#1!50!black] (0,0) rectangle (4,#2*0.5);
    \node[right,font=\small\sffamily] at (4.2,#2*0.25) {\textbf{#3}};
    \node[left,font=\tiny\sffamily,text=eps-gray] at (0,#2*0.25) {#4};
  }%
}

% Clasificación de suelos (USCS)
\newcommand{\soilclass}[2]{%
  \tikz[baseline=(text.base)]{%
    \node[fill=soil-color!20,draw=soil-color,rounded corners=2pt,inner sep=3pt,font=\sffamily\small\bfseries] (text) {#1};
  }%
  \hspace{0.5em}#2%
}

% Caja de ensayo geotécnico
\newtcolorbox{geotestbox}[2][]{%
  colback=white,
  colframe=eps-primary,
  title={\textbf{#2}},
  fonttitle=\small\sffamily,
  boxrule=0.5pt,
  #1
}

% Nivel freático
\newcommand{\watertable}[1]{%
  \begin{center}
  \begin{tikzpicture}
    \fill[soil-color!30] (0,0) rectangle (5,2);
    \fill[water-color!30] (0,0) rectangle (5,#1);
    \draw[water-color,thick,decorate,decoration={snake,amplitude=1pt,segment length=10pt}] (0,#1) -- (5,#1);
    \node[right,font=\small\sffamily] at (5.2,#1) {NF: \SI{#1}{\meter}};
    \draw[<->,thick] (5.5,0) -- (5.5,2) node[midway,right,font=\small] {\SI{2}{\meter}};
  \end{tikzpicture}
  \end{center}
}

% Tabla de sondeo
\newenvironment{boreholedata}[1][Datos de sondeo]{%
  \begin{tcolorbox}[
    colback=white,
    colframe=eps-gray,
    title={\textbf{#1}},
    fonttitle=\small\sffamily,
    boxrule=0.5pt
  ]
  \small\centering
  \begin{tabular}{@{}ccccc@{}}
    \toprule
    \textbf{Prof.} & \textbf{Litología} & \textbf{NSPT} & \textbf{RQD} & \textbf{Muestra} \\
    (m) & & & (\%) & \\
    \midrule
}{%
    \bottomrule
  \end{tabular}
  \end{tcolorbox}
}

% Indicador SPT
\newcommand{\sptvalue}[1]{%
  \tikz[baseline=-0.5ex]{%
    \ifnum#1<10
      \node[fill=eps-danger!20,rounded corners=1pt,inner sep=2pt,font=\small\sffamily] {#1};
    \else\ifnum#1<30
      \node[fill=eps-warning!20,rounded corners=1pt,inner sep=2pt,font=\small\sffamily] {#1};
    \else
      \node[fill=eps-success!20,rounded corners=1pt,inner sep=2pt,font=\small\sffamily] {#1};
    \fi\fi
  }%
}

% Indicador RQD
\newcommand{\rqdvalue}[1]{%
  \tikz[baseline=-0.5ex]{%
    \ifnum#1<25
      \node[fill=eps-danger!20,rounded corners=1pt,inner sep=2pt,font=\small\sffamily] {#1};
    \else\ifnum#1<50
      \node[fill=eps-warning!20,rounded corners=1pt,inner sep=2pt,font=\small\sffamily] {#1};
    \else\ifnum#1<75
      \node[fill=eps-info!20,rounded corners=1pt,inner sep=2pt,font=\small\sffamily] {#1};
    \else
      \node[fill=eps-success!20,rounded corners=1pt,inner sep=2pt,font=\small\sffamily] {#1};
    \fi\fi\fi
  }%
}

% Escala de Mohs
\newcommand{\mohsscale}{%
  \begin{tikzpicture}[scale=0.7]
    \foreach \i/\name in {1/Talco,2/Yeso,3/Calcita,4/Fluorita,5/Apatito,6/Ortoclasa,7/Cuarzo,8/Topacio,9/Corindón,10/Diamante} {
      \pgfmathsetmacro{\col}{(\i-1)*10}
      \fill[eps-primary!\col!white] (\i-1,0) rectangle (\i,0.5);
      \node[below,font=\tiny\sffamily,rotate=45,anchor=east] at (\i-0.5,-0.1) {\name};
      \node[above,font=\tiny\sffamily\bfseries] at (\i-0.5,0.5) {\i};
    }
    \draw (0,0) rectangle (10,0.5);
  \end{tikzpicture}%
}

% Caja de riesgo geológico
\newtcolorbox{hazardbox}[1][]{%
  colback=eps-danger!5,
  colframe=eps-danger,
  title={\textbf{⚠ Riesgo geológico}},
  fonttitle=\small\sffamily,
  boxrule=0.5pt,
  #1
}

% Indicador sísmico
\newcommand{\seismiczone}[1]{%
  \tikz[baseline=(text.base)]{%
    \ifnum#1=1
      \node[fill=eps-success!20,draw=eps-success,rounded corners=2pt,inner sep=3pt,font=\sffamily\small\bfseries] (text) {Zona I - Baja};
    \else\ifnum#1=2
      \node[fill=eps-warning!20,draw=eps-warning,rounded corners=2pt,inner sep=3pt,font=\sffamily\small\bfseries] (text) {Zona II - Media};
    \else
      \node[fill=eps-danger!20,draw=eps-danger,rounded corners=2pt,inner sep=3pt,font=\sffamily\small\bfseries] (text) {Zona III - Alta};
    \fi\fi
  }%
}

% Escala de intensidad sísmica
\newcommand{\seismicintensity}[1]{%
  \tikz[baseline=-0.5ex]{%
    \pgfmathsetmacro{\intensity}{min(#1,10)}
    \foreach \i in {1,...,10} {
      \ifnum\i>#1
        \fill[eps-gray!20] (\i*0.25,0) rectangle ++(0.2,0.3);
      \else
        \pgfmathsetmacro{\col}{(\i)*10}
        \fill[eps-danger!\col!eps-warning] (\i*0.25,0) rectangle ++(0.2,0.3);
      \fi
    }
  }%
}

\begin{document}

\title{\bfseries Componentes de Geología\\[0.3em]\large Ejemplos de uso}
\author{eps-geologia.sty}
\date{\today}
\maketitle

\section{Tipos de rocas}

\begin{center}
\igneousbadge \quad \sedimentarybadge \quad \metamorphicbadge
\end{center}

\section{Ficha de roca}

\begin{rockcard}{igneous-color}
  \textbf{Granito}
  
  Roca ígnea plutónica de textura fanerítica.
  
  \begin{tabular}{@{}ll@{}}
    Composición: & Cuarzo, feldespato, mica \\
    Color: & Gris, rosa, blanco \\
    Dureza Mohs: & 6-7 \\
    Densidad: & \SI{2.65}{\gram\per\cubic\centi\meter} \\
    Porosidad: & 0.5-1.5\% \\
  \end{tabular}
\end{rockcard}

\begin{rockcard}{sedimentary-color}
  \textbf{Caliza}
  
  Roca sedimentaria carbonatada.
  
  \begin{tabular}{@{}ll@{}}
    Composición: & Calcita (\ch{CaCO3}) \\
    Origen: & Biogénico/Químico \\
    Dureza Mohs: & 3 \\
    Densidad: & \SI{2.71}{\gram\per\cubic\centi\meter} \\
    Reacción HCl: & Efervescencia \\
  \end{tabular}
\end{rockcard}

\section{Propiedades geotécnicas}

\begin{geoprops}[Propiedades del suelo]
  Humedad natural: & 18.5\% \\
  Límite líquido: & 42\% \\
  Límite plástico: & 21\% \\
  Índice de plasticidad: & 21\% \\
  Peso específico: & \SI{27.2}{\kilo\newton\per\cubic\meter} \\
  Cohesión: & \SI{25}{\kilo\pascal} \\
  Ángulo de fricción: & 28° \\
\end{geoprops}

\section{Clasificación de suelos USCS}

\soilclass{GW}{Grava bien graduada}

\soilclass{SP}{Arena mal graduada}

\soilclass{CL}{Arcilla de baja plasticidad}

\soilclass{MH}{Limo de alta plasticidad}

\section{Ensayo geotécnico}

\begin{geotestbox}{Ensayo de penetración estándar (SPT)}
  \begin{tabular}{@{}ll@{}}
    Profundidad: & \SI{3.0}{\meter} \\
    N$_{15}$: & 8 golpes \\
    N$_{30}$: & 12 golpes \\
    N$_{45}$: & 10 golpes \\
    \midrule
    N$_{SPT}$: & 22 golpes \\
    Clasificación: & Suelo medianamente denso \\
  \end{tabular}
\end{geotestbox}

\section{Indicadores SPT y RQD}

\begin{tabular}{@{}lcc@{}}
  \toprule
  Descripción & SPT & RQD \\
  \midrule
  Muy blando/Muy malo & \sptvalue{5} & \rqdvalue{15} \\
  Blando/Malo & \sptvalue{15} & \rqdvalue{35} \\
  Medio/Regular & \sptvalue{25} & \rqdvalue{55} \\
  Denso/Bueno & \sptvalue{40} & \rqdvalue{80} \\
  Muy denso/Excelente & \sptvalue{60} & \rqdvalue{95} \\
  \bottomrule
\end{tabular}

\section{Datos de sondeo}

\begin{boreholedata}[Sondeo S-1]
  0.0-1.5 & Relleno antrópico & --- & --- & --- \\
  1.5-3.0 & Arcilla limosa & \sptvalue{8} & --- & MI-1 \\
  3.0-5.0 & Arena fina & \sptvalue{22} & --- & SPT-1 \\
  5.0-8.0 & Grava arenosa & \sptvalue{35} & --- & SPT-2 \\
  8.0-10.0 & Caliza fracturada & R & \rqdvalue{45} & TI-1 \\
  10.0-15.0 & Caliza masiva & R & \rqdvalue{85} & TI-2 \\
\end{boreholedata}

\section{Nivel freático}

\watertable{0.8}

\section{Escala de Mohs}

\begin{center}
\mohsscale
\end{center}

\section{Riesgo geológico}

\begin{hazardbox}
  \textbf{Riesgo de deslizamiento}
  
  Zona con pendiente superior a 30° sobre materiales arcillosos.
  
  \begin{tabular}{@{}ll@{}}
    Factor de seguridad: & 1.15 (< 1.5 mínimo) \\
    Nivel de riesgo: & Alto \\
    Medidas: & Drenaje y contención \\
  \end{tabular}
\end{hazardbox}

\section{Zonificación sísmica}

Peligrosidad sísmica según NCSE-02:

\begin{center}
\seismiczone{1} \quad \seismiczone{2} \quad \seismiczone{3}
\end{center}

\section{Intensidad sísmica}

\begin{tabular}{@{}lc@{}}
  \toprule
  Magnitud & Intensidad \\
  \midrule
  Menor (3.0) & \seismicintensity{2} \\
  Ligero (4.5) & \seismicintensity{4} \\
  Moderado (5.5) & \seismicintensity{6} \\
  Fuerte (6.5) & \seismicintensity{8} \\
  Mayor (7.5) & \seismicintensity{10} \\
  \bottomrule
\end{tabular}

\section{Columna estratigráfica simplificada}

\begin{stratigraphiccolumn}
  \stratlayer{eps-warning}{1}{Cuaternario - Aluvial}{Holoceno}
  
  \vspace{0.2cm}
  
  \stratlayer{sedimentary-color}{1.5}{Margas y calizas}{Mioceno}
  
  \vspace{0.2cm}
  
  \stratlayer{eps-info}{2}{Calizas masivas}{Cretácico}
  
  \vspace{0.2cm}
  
  \stratlayer{metamorphic-color}{1}{Esquistos}{Paleozoico}
\end{stratigraphiccolumn}

\end{document}
