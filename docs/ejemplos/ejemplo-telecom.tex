% =============================================================================
% Ejemplo autocontenido de componentes de Telecomunicaciones
% Compilar con: lualatex -shell-escape ejemplo-telecom.tex
% =============================================================================
\documentclass[12pt,a4paper]{article}

% Configuración básica
\usepackage[spanish]{babel}
\usepackage{fontspec}
\usepackage{xcolor}
\usepackage{tcolorbox}
\tcbuselibrary{skins,breakable}
\usepackage{booktabs}
\usepackage{siunitx}
\usepackage{circuitikz}
\usepackage{tikz}
\usetikzlibrary{positioning,shapes,arrows.meta}

% Definición de colores de la paleta
\definecolor{eps-primary}{HTML}{2563EB}
\definecolor{eps-secondary}{HTML}{7C3AED}
\definecolor{eps-success}{HTML}{059669}
\definecolor{eps-warning}{HTML}{D97706}
\definecolor{eps-danger}{HTML}{DC2626}
\definecolor{eps-info}{HTML}{0284C7}
\definecolor{eps-dark}{HTML}{1F2937}
\definecolor{eps-gray}{HTML}{6B7280}
\definecolor{eps-light}{HTML}{F3F4F6}

% Colores específicos de telecomunicaciones
\definecolor{rf-color}{HTML}{3B82F6}
\definecolor{fiber-color}{HTML}{F59E0B}
\definecolor{copper-color}{HTML}{B45309}
\definecolor{wireless-color}{HTML}{8B5CF6}
\definecolor{satellite-color}{HTML}{10B981}

% --- COMPONENTES DE TELECOMUNICACIONES ---

% Especificaciones de frecuencia
\newcommand{\freqspec}[3]{%
  \begin{tcolorbox}[
    colback=rf-color!5,
    colframe=rf-color,
    title={\textbf{#1}},
    fonttitle=\small\sffamily,
    boxrule=0.5pt
  ]
    \small
    \begin{tabular}{@{}ll@{}}
      \textbf{Banda:} & #2 \\
      \textbf{Frecuencia:} & #3 \\
    \end{tabular}
  \end{tcolorbox}%
}

% Especificación de enlace
\newtcolorbox{linkbudget}[1][]{%
  colback=eps-light,
  colframe=eps-primary,
  title={\textbf{Balance de enlace}},
  fonttitle=\small\sffamily,
  boxrule=0.5pt,
  #1
}

% Tabla de parámetros de señal
\newenvironment{signalparams}[1][Parámetros de señal]{%
  \begin{tcolorbox}[
    colback=white,
    colframe=eps-dark,
    title={\textbf{#1}},
    fonttitle=\small\sffamily,
    boxrule=0.5pt
  ]
  \small\centering
  \begin{tabular}{@{}lcc@{}}
    \toprule
    \textbf{Parámetro} & \textbf{Valor} & \textbf{Unidad} \\
    \midrule
}{%
    \bottomrule
  \end{tabular}
  \end{tcolorbox}
}

% Indicadores de tecnología inalámbrica
\newcommand{\techbadge}[2][eps-primary]{%
  \tikz[baseline=(text.base)]{%
    \node[fill=#1!15,draw=#1,rounded corners=2pt,inner sep=3pt,font=\sffamily\footnotesize\bfseries,text=#1] (text) {#2};
  }%
}

\newcommand{\wifi}{\techbadge[wireless-color]{Wi-Fi}}
\newcommand{\bluetooth}{\techbadge[rf-color]{BT}}
\newcommand{\lte}{\techbadge[eps-success]{LTE}}
\newcommand{\fiveG}{\techbadge[eps-danger]{5G}}
\newcommand{\nbiot}{\techbadge[eps-info]{NB-IoT}}
\newcommand{\lorawan}{\techbadge[eps-secondary]{LoRa}}

% Caja de espectro
\newtcolorbox{spectrumbox}[1][]{%
  colback=rf-color!5,
  colframe=rf-color,
  title={\textbf{Asignación espectral}},
  fonttitle=\small\sffamily,
  boxrule=0.5pt,
  #1
}

% Indicador de nivel de señal
\newcommand{\signallevel}[1]{%
  \tikz[baseline=-0.5ex]{%
    \foreach \i in {1,...,5} {%
      \ifnum\i>#1
        \fill[eps-gray!30] (\i*0.15,0) rectangle ++(0.1,\i*0.08);
      \else
        \ifnum\i<3
          \fill[eps-danger] (\i*0.15,0) rectangle ++(0.1,\i*0.08);
        \else
          \ifnum\i<4
            \fill[eps-warning] (\i*0.15,0) rectangle ++(0.1,\i*0.08);
          \else
            \fill[eps-success] (\i*0.15,0) rectangle ++(0.1,\i*0.08);
          \fi
        \fi
      \fi
    }
  }%
}

% Diagrama de bloques de RF simple
\newenvironment{rfblockdiagram}{%
  \begin{center}
  \begin{tikzpicture}[
    block/.style={rectangle, draw=eps-primary, fill=eps-primary!10, 
                  minimum height=1cm, minimum width=2cm, font=\small\sffamily},
    arrow/.style={-{Stealth[scale=0.8]}, thick, eps-dark}
  ]
}{%
  \end{tikzpicture}
  \end{center}
}

\newcommand{\rfblock}[3][]{\node[block,#1] (#2) {#3};}
\newcommand{\rfconnect}[3][]{\draw[arrow,#1] (#2) -- (#3);}

% Antena badge
\newcommand{\antennabadge}[2]{%
  \begin{tcolorbox}[
    colback=satellite-color!10,
    colframe=satellite-color,
    width=4cm,
    boxrule=0.5pt,
    left=2pt,right=2pt,top=2pt,bottom=2pt
  ]
    \centering\small\sffamily
    \textbf{#1}\\
    Ganancia: #2
  \end{tcolorbox}%
}

% Tabla de canales
\newenvironment{channeltable}[1][Asignación de canales]{%
  \begin{tcolorbox}[
    colback=white,
    colframe=fiber-color,
    title={\textbf{#1}},
    fonttitle=\small\sffamily,
    boxrule=0.5pt
  ]
  \small\centering
  \begin{tabular}{@{}cccc@{}}
    \toprule
    \textbf{Canal} & \textbf{Frecuencia} & \textbf{Ancho} & \textbf{Estado} \\
    \midrule
}{%
    \bottomrule
  \end{tabular}
  \end{tcolorbox}
}

% Estado de canal
\newcommand{\channelfree}{\textcolor{eps-success}{\textbullet\ Libre}}
\newcommand{\channelused}{\textcolor{eps-danger}{\textbullet\ Ocupado}}
\newcommand{\channelreserved}{\textcolor{eps-warning}{\textbullet\ Reservado}}

% Caja de fibra óptica
\newtcolorbox{fiberbox}[1][]{%
  colback=fiber-color!5,
  colframe=fiber-color,
  title={\textbf{Enlace de fibra óptica}},
  fonttitle=\small\sffamily,
  boxrule=0.5pt,
  #1
}

\begin{document}

\title{\bfseries Componentes de Telecomunicaciones\\[0.3em]\large Ejemplos de uso}
\author{eps-telecom.sty}
\date{\today}
\maketitle

\section{Especificaciones de frecuencia}

\freqspec{Enlace de subida}{Banda L}{\SI{1626.5}{\mega\hertz} -- \SI{1660.5}{\mega\hertz}}

\freqspec{Wi-Fi 6E}{6 GHz}{\SI{5.925}{\giga\hertz} -- \SI{7.125}{\giga\hertz}}

\section{Badges de tecnología}

Tecnologías inalámbricas disponibles:

\begin{center}
\wifi \quad \bluetooth \quad \lte \quad \fiveG \quad \nbiot \quad \lorawan
\end{center}

\section{Nivel de señal}

\begin{tabular}{lc}
  Señal muy débil: & \signallevel{1} \\
  Señal débil: & \signallevel{2} \\
  Señal media: & \signallevel{3} \\
  Señal buena: & \signallevel{4} \\
  Señal excelente: & \signallevel{5} \\
\end{tabular}

\section{Parámetros de señal}

\begin{signalparams}[Modulación QPSK]
  Tasa de símbolos & 27.5 & Msps \\
  Tasa de bits & 55 & Mbps \\
  FEC & 3/4 & --- \\
  Eb/N0 requerido & 4.5 & dB \\
\end{signalparams}

\section{Balance de enlace}

\begin{linkbudget}
  \begin{tabular}{@{}lr@{}}
    \textbf{Potencia de transmisión:} & \SI{20}{\dBm} \\
    \textbf{Ganancia antena TX:} & \SI{12}{\dBi} \\
    \textbf{Pérdidas de propagación:} & \SI{-110}{\dB} \\
    \textbf{Ganancia antena RX:} & \SI{15}{\dBi} \\
    \textbf{Pérdidas del receptor:} & \SI{-2}{\dB} \\
    \midrule
    \textbf{Potencia recibida:} & \SI{-65}{\dBm} \\
    \textbf{Sensibilidad:} & \SI{-90}{\dBm} \\
    \textbf{Margen de enlace:} & \SI{25}{\dB} \\
  \end{tabular}
\end{linkbudget}

\section{Asignación espectral}

\begin{spectrumbox}
  \begin{tabular}{@{}lll@{}}
    \toprule
    \textbf{Banda} & \textbf{Uso} & \textbf{Regulación} \\
    \midrule
    700 MHz & LTE/5G & Licenciado \\
    2.4 GHz & Wi-Fi/BT & ISM \\
    5 GHz & Wi-Fi & Sin licencia \\
    28 GHz & 5G mmWave & Licenciado \\
    \bottomrule
  \end{tabular}
\end{spectrumbox}

\section{Antenas}

\begin{center}
\antennabadge{Dipolo}{2.15 dBi}
\hspace{1cm}
\antennabadge{Yagi}{12 dBi}
\hspace{1cm}
\antennabadge{Parabólica}{35 dBi}
\end{center}

\section{Asignación de canales}

\begin{channeltable}[Canales Wi-Fi 2.4 GHz]
  1 & 2412 MHz & 22 MHz & \channelused \\
  6 & 2437 MHz & 22 MHz & \channelfree \\
  11 & 2462 MHz & 22 MHz & \channelreserved \\
\end{channeltable}

\section{Enlace de fibra óptica}

\begin{fiberbox}
  \begin{tabular}{@{}ll@{}}
    \textbf{Tipo de fibra:} & Monomodo G.652.D \\
    \textbf{Longitud de onda:} & \SI{1550}{\nano\meter} \\
    \textbf{Atenuación:} & \SI{0.25}{\dB\per\kilo\meter} \\
    \textbf{Distancia:} & \SI{40}{\kilo\meter} \\
    \textbf{Pérdidas totales:} & \SI{10}{\dB} + \SI{3}{\dB} (conectores) \\
  \end{tabular}
\end{fiberbox}

\section{Diagrama de bloques RF}

\begin{rfblockdiagram}
  \rfblock{ant}{Antena}
  \rfblock[right=1.5cm of ant]{lna}{LNA}
  \rfblock[right=1.5cm of lna]{mix}{Mezclador}
  \rfblock[right=1.5cm of mix]{adc}{ADC}
  
  \rfconnect{ant}{lna}
  \rfconnect{lna}{mix}
  \rfconnect{mix}{adc}
\end{rfblockdiagram}

\section{Circuito con CircuiTikZ}

\begin{center}
\begin{circuitikz}[scale=0.8, transform shape]
  \draw (0,0) to[sinusoidal voltage source, l=$V_{in}$] (0,3)
        to[R, l=$R_1$] (3,3)
        to[C, l=$C_1$] (3,0) -- (0,0);
  \draw (3,3) to[short, -o] (4,3) node[right] {$V_{out}$};
  \draw (3,0) to[short, -o] (4,0) node[right] {GND};
\end{circuitikz}
\end{center}

\end{document}
