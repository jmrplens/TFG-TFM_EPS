% ============================================================================
% EJEMPLO: Componentes de Software (eps-software)
% ============================================================================
% Este archivo muestra los componentes disponibles en eps-software.sty
% Compilar con: lualatex ejemplo-software.tex
% ============================================================================

\documentclass[11pt,a4paper]{article}

% Idioma y fuentes
\usepackage[spanish]{babel}
\usepackage{fontspec}

% Configurar rutas para encontrar los paquetes
\makeatletter
\def\input@path{{../../sty/}{../../sty/componentes/}}
\makeatother

% Cargar componentes de software
\usepackage[software]{eps-componentes}

% Geometría de página
\usepackage[margin=2.5cm]{geometry}

% Para tablas largas
\usepackage{longtable}

% Título del documento
\title{\bfseries Componentes de Software\\[0.5em]\large eps-software.sty}
\author{Plantilla TFG/TFM EPS UA}
\date{\today}

\begin{document}

\maketitle
\tableofcontents
\newpage

% ============================================================================
\section{API REST}
% ============================================================================

\subsection{Métodos HTTP}

Métodos disponibles: \httpmethod{GET} \quad \httpmethod{POST} \quad \httpmethod{PUT} \quad \httpmethod{PATCH} \quad \httpmethod{DELETE}

\subsection{Endpoint Completo}

\begin{apiendpoint}{GET}{/api/users/:id}
  \apidescription{Obtiene la información de un usuario por su ID.}
  
  \apiparams{
    id & string & Identificador del usuario & Sí \\
  }
  
  \apiresponse{200}{json}{\{"id": "123", "name": "Juan"\}}
\end{apiendpoint}

\begin{apiendpoint}{POST}{/api/users}
  \apidescription{Crea un nuevo usuario en el sistema.}
  
  \apiresponse{201}{json}{\{"id": "124", "created": true\}}
\end{apiendpoint}

% ============================================================================
\section{Terminal y Comandos}
% ============================================================================

\subsection{Terminal Básica}

\begin{terminal}
  \prompt npm install\\
  \prompt npm run build\\
  \termcomment{Compilación exitosa}
\end{terminal}

\subsection{Terminal con Título}

\begin{terminal}[Servidor de Producción]
  \promptroot systemctl start nginx\\
  \promptroot systemctl enable nginx\\
  \termcomment{nginx.service enabled}
\end{terminal}

% ============================================================================
\section{Diagramas UML}
% ============================================================================

\subsection{Clase UML}

\begin{umlclass}{Usuario}{%
  \visibility{-}id: int \\
  \visibility{-}nombre: string \\
  \visibility{-}email: string%
}{%
  \visibility{+}getNombre(): string \\
  \visibility{+}setNombre(n: string): void \\
  \visibility{#}validarEmail(): bool%
}

\subsection{Interface UML}

\begin{umlinterface}{IRepository}{%
  \visibility{+}find(id: int): Entity \\
  \visibility{+}save(e: Entity): void \\
  \visibility{+}delete(id: int): bool%
}
\end{umlinterface}

% ============================================================================
\section{Comandos Git}
% ============================================================================

Branch: \gitbranch{main} \quad \gitbranch{develop}

Tag: \gittag{v1.0.0} \quad \gittag{v2.0.0}

Commit:

\gitcommit{abc1234}{feat: nueva funcionalidad}{Juan García}{2024-01-15}

\gitcommit{def5678}{fix: corrección de bug}{María López}{2024-01-16}

% ============================================================================
\section{Base de Datos}
% ============================================================================

\dbtable{usuarios}{
  \pkicon~id & INT & AUTO\_INCREMENT \\
  nombre & VARCHAR(100) & NOT NULL \\
  email & VARCHAR(255) & UNIQUE \\
  \fkicon~rol\_id & INT & FOREIGN KEY \\
}

% ============================================================================
\section{Logs}
% ============================================================================

\begin{logbox}
  \logentry{INFO}{2024-01-15 10:30:45}{Servidor iniciado en puerto 3000}
  \logentry{WARN}{2024-01-15 10:31:00}{Memoria al 80\%}
  \logentry{ERROR}{2024-01-15 10:31:15}{Conexión a BD fallida}
  \logentry{DEBUG}{2024-01-15 10:31:20}{Reintentando conexión...}
\end{logbox}

% ============================================================================
\section{Configuración}
% ============================================================================

\begin{configbox}
  \envvar{DATABASE\_URL}{postgresql://localhost/db}\\
  \envvar{API\_KEY}{sk-xxxxxxxxxxxxx}\\
  \envvar{DEBUG}{true}
\end{configbox}

\end{document}
