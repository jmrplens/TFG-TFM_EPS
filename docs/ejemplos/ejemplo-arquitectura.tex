% =============================================================================
% Ejemplo autocontenido de componentes de Arquitectura/Construcción
% Compilar con: lualatex -shell-escape ejemplo-arquitectura.tex
% =============================================================================
\documentclass[12pt,a4paper]{article}

% Configuración básica
\usepackage[spanish]{babel}
\usepackage{fontspec}
\usepackage{xcolor}
\usepackage{tcolorbox}
\tcbuselibrary{skins,breakable}
\usepackage{booktabs}
\usepackage{siunitx}
\usepackage{tikz}
\usetikzlibrary{positioning,shapes,patterns,calc}

% Definición de colores de la paleta
\definecolor{eps-primary}{HTML}{2563EB}
\definecolor{eps-secondary}{HTML}{7C3AED}
\definecolor{eps-success}{HTML}{059669}
\definecolor{eps-warning}{HTML}{D97706}
\definecolor{eps-danger}{HTML}{DC2626}
\definecolor{eps-info}{HTML}{0284C7}
\definecolor{eps-dark}{HTML}{1F2937}
\definecolor{eps-gray}{HTML}{6B7280}
\definecolor{eps-light}{HTML}{F3F4F6}

% Colores específicos de arquitectura
\definecolor{concrete-color}{HTML}{6B7280}
\definecolor{steel-color}{HTML}{3B82F6}
\definecolor{wood-color}{HTML}{92400E}
\definecolor{brick-color}{HTML}{DC2626}
\definecolor{glass-color}{HTML}{06B6D4}

% --- COMPONENTES DE ARQUITECTURA ---

% Ficha de material
\newtcolorbox{materialcard}[2][]{%
  colback=white,
  colframe=#2,
  title={\textbf{Ficha de material}},
  fonttitle=\small\sffamily,
  boxrule=0.5pt,
  left=4pt,
  right=4pt,
  #1
}

% Indicador de material
\newcommand{\materialbadge}[2]{%
  \tikz[baseline=(text.base)]{%
    \node[fill=#2!15,draw=#2,rounded corners=2pt,inner sep=3pt,font=\sffamily\footnotesize\bfseries,text=#2] (text) {#1};
  }%
}

\newcommand{\concretebadge}{\materialbadge{Hormigón}{concrete-color}}
\newcommand{\steelbadge}{\materialbadge{Acero}{steel-color}}
\newcommand{\woodbadge}{\materialbadge{Madera}{wood-color}}
\newcommand{\brickbadge}{\materialbadge{Ladrillo}{brick-color}}
\newcommand{\glassbadge}{\materialbadge{Vidrio}{glass-color}}

% Propiedades de material
\newenvironment{materialprops}[1][Propiedades]{%
  \begin{tcolorbox}[
    colback=eps-light,
    colframe=eps-dark,
    title={\textbf{#1}},
    fonttitle=\small\sffamily,
    boxrule=0.5pt
  ]
  \small
  \begin{tabular}{@{}ll@{}}
}{%
  \end{tabular}
  \end{tcolorbox}
}

% Caja de normativa
\newtcolorbox{normativabox}[1][]{%
  colback=eps-info!5,
  colframe=eps-info,
  title={\textbf{Referencia normativa}},
  fonttitle=\small\sffamily,
  boxrule=0.5pt,
  #1
}

% Caja de cálculo estructural
\newtcolorbox{structuralbox}[1][]{%
  colback=steel-color!5,
  colframe=steel-color,
  title={\textbf{Cálculo estructural}},
  fonttitle=\small\sffamily,
  boxrule=0.5pt,
  #1
}

% Detalle constructivo
\newtcolorbox{constructiondetail}[2][]{%
  colback=white,
  colframe=eps-dark,
  title={\textbf{#2}},
  fonttitle=\small\sffamily,
  boxrule=0.5pt,
  #1
}

% Tabla de capas
\newenvironment{layertable}[1][Composición de cerramiento]{%
  \begin{tcolorbox}[
    colback=white,
    colframe=eps-gray,
    title={\textbf{#1}},
    fonttitle=\small\sffamily,
    boxrule=0.5pt
  ]
  \small\centering
  \begin{tabular}{@{}lccc@{}}
    \toprule
    \textbf{Capa} & \textbf{Espesor} & \textbf{$\lambda$} & \textbf{R} \\
    & (mm) & (W/m·K) & (m²·K/W) \\
    \midrule
}{%
    \bottomrule
  \end{tabular}
  \end{tcolorbox}
}

% Indicador de transmitancia
\newcommand{\transmitanciaU}[2]{%
  \begin{tcolorbox}[
    colback=white,
    colframe=eps-success,
    width=4cm,
    boxrule=0.5pt,
    left=2pt,right=2pt,top=2pt,bottom=2pt
  ]
    \centering\small\sffamily
    \textbf{U = #1}\\
    Límite: #2
  \end{tcolorbox}%
}

% Verificación CTE
\newcommand{\cteok}{\textcolor{eps-success}{\textbf{✓ Cumple CTE}}}
\newcommand{\ctefail}{\textcolor{eps-danger}{\textbf{✗ No cumple}}}
\newcommand{\ctewarn}{\textcolor{eps-warning}{\textbf{⚠ Revisar}}}

% Caja de certificación energética
\newcommand{\energyrating}[1]{%
  \tikz[baseline=(char.base)]{%
    \ifnum\pdfstrcmp{#1}{A}=0
      \node[fill=green!80!black,text=white,font=\bfseries\sffamily,minimum width=1.5em,minimum height=1.5em,rounded corners=2pt] (char) {A};
    \else\ifnum\pdfstrcmp{#1}{B}=0
      \node[fill=green!60!yellow,text=white,font=\bfseries\sffamily,minimum width=1.5em,minimum height=1.5em,rounded corners=2pt] (char) {B};
    \else\ifnum\pdfstrcmp{#1}{C}=0
      \node[fill=yellow!80!green,text=black,font=\bfseries\sffamily,minimum width=1.5em,minimum height=1.5em,rounded corners=2pt] (char) {C};
    \else\ifnum\pdfstrcmp{#1}{D}=0
      \node[fill=yellow,text=black,font=\bfseries\sffamily,minimum width=1.5em,minimum height=1.5em,rounded corners=2pt] (char) {D};
    \else\ifnum\pdfstrcmp{#1}{E}=0
      \node[fill=orange,text=white,font=\bfseries\sffamily,minimum width=1.5em,minimum height=1.5em,rounded corners=2pt] (char) {E};
    \else\ifnum\pdfstrcmp{#1}{F}=0
      \node[fill=orange!80!red,text=white,font=\bfseries\sffamily,minimum width=1.5em,minimum height=1.5em,rounded corners=2pt] (char) {F};
    \else
      \node[fill=red,text=white,font=\bfseries\sffamily,minimum width=1.5em,minimum height=1.5em,rounded corners=2pt] (char) {G};
    \fi\fi\fi\fi\fi\fi
  }%
}

% Medición de obra
\newenvironment{measurementtable}[1][Medición]{%
  \begin{tcolorbox}[
    colback=white,
    colframe=eps-primary,
    title={\textbf{#1}},
    fonttitle=\small\sffamily,
    boxrule=0.5pt
  ]
  \small\centering
  \begin{tabular}{@{}p{5cm}ccc@{}}
    \toprule
    \textbf{Partida} & \textbf{Ud.} & \textbf{Cantidad} & \textbf{Precio} \\
    \midrule
}{%
    \bottomrule
  \end{tabular}
  \end{tcolorbox}
}

% Sección constructiva simplificada
\newcommand{\wallsection}[1]{%
  \begin{center}
  \begin{tikzpicture}[scale=0.8]
    % Fondo
    \fill[eps-light] (0,0) rectangle (4,5);
    
    % Capas ejemplo
    \fill[brick-color!30] (0,0) rectangle (0.5,5);
    \fill[yellow!30] (0.5,0) rectangle (1.5,5);
    \fill[concrete-color!30] (1.5,0) rectangle (2.5,5);
    \fill[yellow!30] (2.5,0) rectangle (3.5,5);
    \fill[white] (3.5,0) rectangle (4,5);
    
    % Líneas de separación
    \draw[eps-dark] (0,0) rectangle (4,5);
    \draw[eps-dark] (0.5,0) -- (0.5,5);
    \draw[eps-dark] (1.5,0) -- (1.5,5);
    \draw[eps-dark] (2.5,0) -- (2.5,5);
    \draw[eps-dark] (3.5,0) -- (3.5,5);
    
    % Etiquetas
    \node[font=\tiny\sffamily,rotate=90] at (0.25,2.5) {Ladrillo};
    \node[font=\tiny\sffamily,rotate=90] at (1,2.5) {Aislamiento};
    \node[font=\tiny\sffamily,rotate=90] at (2,2.5) {Hormigón};
    \node[font=\tiny\sffamily,rotate=90] at (3,2.5) {Aislamiento};
    \node[font=\tiny\sffamily,rotate=90] at (3.75,2.5) {Yeso};
    
    % Título
    \node[below,font=\small\sffamily\bfseries] at (2,-0.3) {#1};
  \end{tikzpicture}
  \end{center}
}

\begin{document}

\title{\bfseries Componentes de Arquitectura\\[0.3em]\large Ejemplos de uso}
\author{eps-arquitectura.sty}
\date{\today}
\maketitle

\section{Badges de materiales}

Materiales estructurales principales:

\begin{center}
\concretebadge \quad \steelbadge \quad \woodbadge \quad \brickbadge \quad \glassbadge
\end{center}

\section{Ficha de material}

\begin{materialcard}{concrete-color}
  \textbf{Hormigón armado HA-25/B/20/IIa}
  
  \begin{tabular}{@{}ll@{}}
    Resistencia característica: & \SI{25}{\mega\pascal} \\
    Consistencia: & Blanda (B) \\
    Tamaño máximo árido: & \SI{20}{\milli\meter} \\
    Ambiente: & IIa (humedad alta) \\
    Recubrimiento mínimo: & \SI{35}{\milli\meter} \\
  \end{tabular}
\end{materialcard}

\section{Propiedades de material}

\begin{materialprops}[Acero B500SD]
  Límite elástico: & \SI{500}{\mega\pascal} \\
  Resistencia a tracción: & \SI{575}{\mega\pascal} \\
  Alargamiento rotura: & $\geq$ 16\% \\
  Módulo de elasticidad: & \SI{200}{\giga\pascal} \\
\end{materialprops}

\section{Referencia normativa}

\begin{normativabox}
  \textbf{CTE DB-HE} -- Documento Básico de Ahorro de Energía
  
  Sección HE1: Limitación de la demanda energética
  
  Zona climática: D3 (Madrid)
  
  \textbf{Valores límite de transmitancia térmica:}
  \begin{itemize}
    \item Muros: U $\leq$ \SI{0.41}{\watt\per\square\meter\per\kelvin}
    \item Cubiertas: U $\leq$ \SI{0.35}{\watt\per\square\meter\per\kelvin}
    \item Huecos: U $\leq$ \SI{2.30}{\watt\per\square\meter\per\kelvin}
  \end{itemize}
\end{normativabox}

\section{Cálculo estructural}

\begin{structuralbox}
  \textbf{Dimensionado de viga de hormigón armado}
  
  \begin{tabular}{@{}ll@{}}
    Momento flector: & $M_d = \SI{150}{\kilo\newton\meter}$ \\
    Canto útil: & $d = \SI{450}{\milli\meter}$ \\
    Ancho: & $b = \SI{300}{\milli\meter}$ \\
    \midrule
    Momento reducido: & $\mu = \dfrac{M_d}{b \cdot d^2 \cdot f_{cd}} = 0.185$ \\
    Cuantía mecánica: & $\omega = 0.212$ \\
    Armadura necesaria: & $A_s = \SI{8.52}{\centi\meter\squared}$ \\
    \midrule
    \textbf{Armado:} & 4$\phi$16 + 2$\phi$12 \\
  \end{tabular}
\end{structuralbox}

\section{Detalle constructivo}

\begin{constructiondetail}{Encuentro fachada-forjado}
  \begin{itemize}
    \item Hoja exterior de ladrillo cara vista
    \item Aislamiento térmico continuo
    \item Rotura de puente térmico en frente de forjado
    \item Sellado perimetral con banda compresible
  \end{itemize}
\end{constructiondetail}

\section{Composición de cerramiento}

\begin{layertable}[Fachada ventilada]
  Aplacado piedra & 30 & 2.30 & 0.013 \\
  Cámara ventilada & 40 & --- & 0.090 \\
  Aislamiento MW & 80 & 0.035 & 2.286 \\
  Bloque hormigón & 200 & 0.95 & 0.211 \\
  Enlucido yeso & 15 & 0.57 & 0.026 \\
  \midrule
  \multicolumn{3}{r}{\textbf{Total R:}} & \textbf{2.626} \\
\end{layertable}

\section{Transmitancia térmica}

\begin{center}
\transmitanciaU{\SI{0.38}{\watt\per\square\meter\per\kelvin}}{\SI{0.41}{\watt\per\square\meter\per\kelvin}}
\hspace{2cm}
\cteok
\end{center}

Otros estados de verificación:

\cteok \quad \ctefail \quad \ctewarn

\section{Certificación energética}

Calificaciones energéticas: \energyrating{A} \energyrating{B} \energyrating{C} \energyrating{D} \energyrating{E} \energyrating{F} \energyrating{G}

\section{Medición de obra}

\begin{measurementtable}[Capítulo 04: Estructura]
  Hormigón HA-25 en pilares & m³ & 12.50 & 125.00 € \\
  Acero B500SD en pilares & kg & 1875 & 1.20 € \\
  Encofrado metálico pilares & m² & 62.50 & 18.00 € \\
\end{measurementtable}

\section{Sección constructiva}

\wallsection{Fachada con aislamiento por el exterior}

\end{document}
