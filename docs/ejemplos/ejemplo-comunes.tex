% ============================================================================
% EJEMPLO: Componentes Comunes (eps-comunes)
% ============================================================================
% Este archivo muestra todos los componentes disponibles en eps-comunes.sty
% Compilar con: lualatex ejemplo-comunes.tex
% ============================================================================

\documentclass[11pt,a4paper]{article}

% Idioma y fuentes
\usepackage[spanish]{babel}
\usepackage{fontspec}

% Configurar rutas para encontrar los paquetes
\makeatletter
\def\input@path{{../../sty/}{../../sty/componentes/}}
\makeatother

% Cargar componentes comunes
\usepackage[comunes]{eps-componentes}

% Geometría de página
\usepackage[margin=2.5cm]{geometry}

% Título del documento
\title{\bfseries Componentes Comunes\\[0.5em]\large eps-comunes.sty}
\author{Plantilla TFG/TFM EPS UA}
\date{\today}

\begin{document}

\maketitle
\tableofcontents
\newpage

% ============================================================================
\section{Cajas de Información}
% ============================================================================

Las cajas de información permiten destacar contenido importante con diferentes niveles de énfasis.

\subsection{Caja de Información}
\begin{infobox}
  Esta es una caja de información general. Úsala para resaltar datos relevantes o explicaciones adicionales que complementen el texto principal.
\end{infobox}

\subsection{Caja de Advertencia}
\begin{warningbox}
  Esta es una caja de advertencia. Ideal para alertar al lector sobre precauciones o consideraciones importantes a tener en cuenta.
\end{warningbox}

\subsection{Caja de Peligro}
\begin{dangerbox}
  Esta es una caja de peligro. Utilízala para señalar errores comunes, situaciones críticas o información que requiere atención inmediata.
\end{dangerbox}

\subsection{Caja de Éxito}
\begin{successbox}
  Esta es una caja de éxito. Perfecta para indicar resultados positivos, soluciones correctas o confirmaciones.
\end{successbox}

\subsection{Caja de Consejo}
\begin{tipbox}
  Esta es una caja de consejo. Úsala para compartir trucos, mejores prácticas o sugerencias útiles.
\end{tipbox}

% ============================================================================
\section{Cajas con Título}
% ============================================================================

\subsection{Caja de Título Básica}
\begin{titlebox}{Definición importante}
  Las cajas con título son ideales para definiciones, teoremas o cualquier contenido que necesite un encabezado destacado.
\end{titlebox}

\subsection{Caja de Definición}
\begin{definitionbox}{Algoritmo}
  Conjunto ordenado y finito de operaciones que permite hallar la solución de un problema.
\end{definitionbox}

\subsection{Caja de Ejemplo}
\begin{examplebox}
  Este es un ejemplo práctico de cómo usar las cajas en tu documento.
\end{examplebox}

\subsection{Caja de Importante}
\begin{importantbox}
  Esta información es crucial y no debe pasarse por alto.
\end{importantbox}

% ============================================================================
\section{Listas Especiales}
% ============================================================================

\subsection{Lista de Verificación (Checklist)}
\begin{checklist}
  \item[\checked] Tarea completada correctamente
  \item[\checked] Segunda tarea también completada
  \item[\unchecked] Tarea pendiente de realizar
  \item[\partialchecked] Tarea parcialmente completada
\end{checklist}

\subsection{Lista de Pros y Contras}
\begin{proscons}
  \pro Ventaja clara del enfoque propuesto
  \pro Otra ventaja importante a destacar
  \pro Tercera ventaja significativa
  \con Desventaja o limitación conocida
  \con Otro aspecto negativo a considerar
\end{proscons}

\subsection{Lista de Pasos}
\begin{steplist}
  \step Preparar el entorno de desarrollo
  \step Instalar las dependencias necesarias
  \step Configurar los parámetros iniciales
  \step Ejecutar las pruebas de verificación
  \step Desplegar el sistema en producción
\end{steplist}

% ============================================================================
\section{Elementos Visuales}
% ============================================================================

\subsection{Badges (Etiquetas)}

Puedes usar badges para etiquetar elementos:
\badge{Nuevo} \badge[eps-success]{Completado} \badge[eps-warning]{En progreso} \badge[eps-danger]{Urgente}

\bigskip
Badges predefinidos: \badgenew{} \badgewip{} \badgedeprecated{} \badgebeta{}

\subsection{Versión}
Versión del software: \version{1.2.3}

\subsection{Barra de Progreso}

Indicadores de progreso visual (valores de 0 a 100):

\noindent 25\%: \progressbar{25}

\noindent 50\%: \progressbar{50}

\noindent 75\%: \progressbar{75}

\noindent 100\%: \progressbar{100}

\subsection{Valoración con Estrellas}

Sistema de rating (valor, máximo): \rating{3}{5} (3 de 5 estrellas)

Otra valoración: \rating{4}{5} (4 de 5 estrellas)

Valoración perfecta: \rating{5}{5}

\subsection{Barra de Nivel}

Nivel de habilidad (valor, máximo): \levelbar{4}{5} (nivel 4 de 5)

Nivel bajo: \levelbar{2}{5} (nivel 2 de 5)

% ============================================================================
\section{Líneas de Tiempo}
% ============================================================================

\begin{timeline}
  \timeitem{2020}{Inicio del proyecto de investigación}
  \timeitem{2021}{Primera fase de desarrollo completada}
  \timeitem{2022}{Publicación de resultados preliminares}
  \timeitem{2023}{Validación experimental del modelo}
  \timeitem{2024}{Defensa del trabajo final}
\end{timeline}

% ============================================================================
\section{Comparaciones}
% ============================================================================

\begin{comparison}{Método A}{Método B}
  \comprow{Velocidad}{Rápido}{Lento}
  \comprow{Memoria}{Alta}{Baja}
  \comprow{Complejidad}{$O(n \log n)$}{$O(n^2)$}
  \comprow{Facilidad}{Media}{Alta}
\end{comparison}

% ============================================================================
\section{Citas Destacadas}
% ============================================================================

\begin{quotebox}[Albert Einstein]
  La imaginación es más importante que el conocimiento. El conocimiento es limitado, mientras que la imaginación abarca todo el mundo.
\end{quotebox}

% ============================================================================
\section{Tarjetas de Información}
% ============================================================================

\subsection{Tarjeta de Persona}

\personcard{Dr. Juan García}{Director de Proyecto}{Experto en inteligencia artificial con más de 10 años de experiencia.}

\subsection{Tarjeta de Estadística}

\noindent
\statcard{1.5M}{Usuarios activos}{\faUsers}{eps-primary}
\statcard{99.9\%}{Disponibilidad}{\faServer}{eps-success}
\statcard{42}{Países}{\faGlobe}{eps-info}

\end{document}
