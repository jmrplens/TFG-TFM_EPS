% Ejemplo de componentes comunes
% Compilar con: lualatex -shell-escape ejemplo-comunes.tex
\documentclass{article}
\usepackage[spanish]{babel}
\usepackage{fontspec}

% Ruta a los paquetes
\makeatletter
\providecommand\input@path{}
\edef\input@path{{../../cls/}{../../sty/}\input@path}
\makeatother

% Cargar solo componentes comunes
\usepackage{eps-componentes}

\begin{document}

\section*{Componentes Comunes}

\subsection*{Cajas de información}

\begin{infobox}
  Esta es una caja de información general.
\end{infobox}

\begin{successbox}
  Esta es una caja de éxito.
\end{successbox}

\begin{warningbox}
  Esta es una caja de advertencia.
\end{warningbox}

\begin{dangerbox}
  Esta es una caja de peligro.
\end{dangerbox}

\begin{tipbox}
  Esta es una caja de consejo.
\end{tipbox}

\subsection*{Cajas con título}

\begin{titlebox}{Mi Título}
  Contenido de la caja.
\end{titlebox}

\begin{definitionbox}{Algoritmo}
  Conjunto ordenado de operaciones.
\end{definitionbox}

\subsection*{Listas especiales}

\begin{checklist}
  \item[\checked] Tarea completada
  \item[\unchecked] Tarea pendiente
\end{checklist}

\begin{proscons}
  \pro Ventaja del sistema
  \con Desventaja menor
\end{proscons}

\begin{steplist}
  \step Primer paso
  \step Segundo paso
\end{steplist}

\subsection*{Badges}

\badge{Estable} \quad
\badge[eps-success]{OK} \quad
\badgenew \quad
\badgewip

\vspace{1em}
Versión: \version{2.1.0}

\subsection*{Indicadores}

Progreso: \progressbar{75}

Rating: \rating{4}{5}

Nivel: \levelbar{3}{5}

\subsection*{Timeline}

\begin{timeline}
  \timeitem{2024}{Inicio del proyecto}
  \timeitem{2025}{Desarrollo principal}
  \timeitem{2026}{Entrega final}
\end{timeline}

\end{document}
