%% =============================================================================
%% CAPÍTULO: DESARROLLO - DEMOSTRACIÓN DE ESTILOS DE CÓDIGO
%% =============================================================================
\chapter{Desarrollo: Estilos de Código}
\label{cap:desarrollo}

Este capítulo demuestra los estilos de cajas de código disponibles en esta plantilla,
basados en el diseño de Visual Studio Code.

%% =============================================================================
\section{Estilo VS Code Light}
\label{sec:vscode-light}

El estilo VS Code Light imita la apariencia del editor Visual Studio Code en su tema claro.

\subsection{Python}

\begin{pythoncode}
def fibonacci(n):
    if n <= 1:
        return n
    return fibonacci(n-1) + fibonacci(n-2)

for i in range(10):
    print(f"F({i}) = {fibonacci(i)}")
\end{pythoncode}

\subsection{Sin numeración de líneas}

Usando el sufijo \texttt{NN} (No Numbers) se elimina la numeración:

\begin{pythoncodeNN}
mensaje = "Hola, mundo"
print(mensaje)
\end{pythoncodeNN}

\subsection{JavaScript}

\begin{jscode}
function validateEmail(email) {
    const regex = /^[a-z]+@[a-z]+\.[a-z]+$/;
    return regex.test(email);
}

async function fetchData(url) {
    const response = await fetch(url);
    return response.json();
}
\end{jscode}

\subsection{Java}

\begin{javacode}
public class HelloWorld {
    public static void main(String[] args) {
        System.out.println("Hola desde Java!");
    }
}
\end{javacode}

%% =============================================================================
\section{Estilo VS Code Dark}
\label{sec:vscode-dark}

El tema oscuro de VS Code, ideal para documentación técnica o presentaciones.

\subsection{Python Dark}

\begin{pythoncodeDark}
import numpy as np

def matrix_multiply(A, B):
    return np.dot(A, B)

A = np.array([[1, 2], [3, 4]])
B = np.array([[5, 6], [7, 8]])
result = matrix_multiply(A, B)
\end{pythoncodeDark}

\subsection{JavaScript Dark}

\begin{jscodeDark}
const express = require('express');
const app = express();

app.get('/api/users', async (req, res) => {
    const users = await User.find();
    res.json(users);
});

app.listen(3000);
\end{jscodeDark}

%% =============================================================================
\section{Lenguajes Web}
\label{sec:web}

\subsection{HTML}

\begin{htmlcode}
<!DOCTYPE html>
<html lang="es">
<head>
    <meta charset="UTF-8">
    <title>Mi Aplicación</title>
</head>
<body>
    <h1>Bienvenido</h1>
</body>
</html>
\end{htmlcode}

\subsection{CSS}

\begin{csscode}
.container {
    max-width: 1200px;
    margin: 0 auto;
    display: grid;
    grid-template-columns: repeat(auto-fit, minmax(300px, 1fr));
    gap: 1rem;
}

.card:hover {
    transform: translateY(-5px);
}
\end{csscode}

%% =============================================================================
\section{Lenguajes de Datos}
\label{sec:datos}

\subsection{SQL}

\begin{sqlcode}
SELECT 
    usuarios.nombre,
    COUNT(pedidos.id) as total_pedidos
FROM usuarios
LEFT JOIN pedidos ON usuarios.id = pedidos.usuario_id
GROUP BY usuarios.id
ORDER BY total_pedidos DESC
LIMIT 10;
\end{sqlcode}

%% =============================================================================
\section{Lenguajes de Sistemas}
\label{sec:sistemas}

\subsection{C++}

\begin{cppcode}
\#include <iostream>
\#include <vector>

int main() {
    std::vector<int> nums = {5, 2, 8, 1, 9};
    for (const auto& n : nums) {
        std::cout << n << " ";
    }
    return 0;
}
\end{cppcode}

\subsection{Rust}

\begin{rustcode}
fn main() {
    let mensaje = "Hola desde Rust";
    println!("{}", mensaje);
    
    let numeros = vec![1, 2, 3, 4, 5];
    for n in numeros {
        println!("{}", n);
    }
}
\end{rustcode}

\subsection{Go}

\begin{gocode}
package main

import "fmt"

func main() {
    fmt.Println("Hola desde Go")
    
    nums := []int{1, 2, 3, 4, 5}
    for _, n := range nums {
        fmt.Println(n)
    }
}
\end{gocode}

%% =============================================================================
\section{Otros Lenguajes}
\label{sec:otros}

\subsection{PHP}

\begin{phpcode}
<?php
class Usuario {
    private $nombre;
    
    public function __construct($nombre) {
        $this->nombre = $nombre;
    }
    
    public function saludar() {
        return "Hola, " . $this->nombre . "!";
    }
}

$usuario = new Usuario("Juan");
echo $usuario->saludar();
?>
\end{phpcode}

\subsection{Ruby}

\begin{rubycode}
class Persona
  attr_accessor :nombre, :edad
  
  def initialize(nombre, edad)
    @nombre = nombre
    @edad = edad
  end
  
  def saludar
    puts "Hola, soy " + @nombre
  end
end

persona = Persona.new("Maria", 25)
persona.saludar
\end{rubycode}

%% =============================================================================
\section{DevOps y Configuración}
\label{sec:devops}

\subsection{Docker}

\begin{dockercode}
FROM node:18-alpine

WORKDIR /app

COPY package*.json ./
RUN npm ci --only=production

COPY . .

EXPOSE 3000

CMD ["node", "server.js"]
\end{dockercode}

%% =============================================================================
\section{Entorno Genérico}
\label{sec:generico}

El entorno \texttt{codigo\{lenguaje\}} permite usar cualquier lenguaje soportado 
por Pygments:

\begin{codigo}{swift}
import Foundation

struct Persona {
    let nombre: String
    let edad: Int
    
    func saludar() -> String {
        return "Hola, soy " + nombre
    }
}

let persona = Persona(nombre: "Carlos", edad: 30)
print(persona.saludar())
\end{codigo}

\subsection{Versión Dark del Genérico}

\begin{codigoDark}{kotlin}
data class Usuario(val nombre: String, val email: String)

fun main() {
    val usuarios = listOf(
        Usuario("Ana", "ana@mail.com"),
        Usuario("Luis", "luis@mail.com")
    )
    
    usuarios.forEach { println(it) }
}
\end{codigoDark}

%% =============================================================================
\section{Resumen de Entornos Disponibles}
\label{sec:resumen-entornos}

\begin{table}[H]
\centering
\caption{Entornos de código con icono disponibles}
\label{tab:entornos-codigo}
\begin{tabular}{llll}
\toprule
\textbf{Entorno} & \textbf{Lenguaje} & \textbf{Icono} & \textbf{Dark} \\
\midrule
\texttt{pythoncode} & Python & \faIcon{python} & \texttt{pythoncodeDark} \\
\texttt{jscode} & JavaScript & \faIcon{js-square} & \texttt{jscodeDark} \\
\texttt{javacode} & Java & \faIcon{java} & \texttt{javacodeDark} \\
\texttt{cppcode} & C++ & \faIcon{copyright} & \texttt{cppcodeDark} \\
\texttt{rustcode} & Rust & \faIcon{rust} & \texttt{rustcodeDark} \\
\texttt{gocode} & Go & \faIcon{google} & \texttt{gocodeDark} \\
\texttt{phpcode} & PHP & \faIcon{php} & \texttt{phpcodeDark} \\
\texttt{rubycode} & Ruby & \faIcon{gem} & \texttt{rubycodeDark} \\
\texttt{htmlcode} & HTML & \faIcon{html5} & \texttt{htmlcodeDark} \\
\texttt{csscode} & CSS & \faIcon{css3-alt} & \texttt{csscodeDark} \\
\texttt{sqlcode} & SQL & \faIcon{database} & \texttt{sqlcodeDark} \\
\texttt{bashcode} & Bash & \faIcon{terminal} & \texttt{bashcodeDark} \\
\texttt{dockercode} & Dockerfile & \faIcon{docker} & \texttt{dockercodeDark} \\
\bottomrule
\end{tabular}
\end{table}

\begin{table}[H]
\centering
\caption{Sufijos disponibles para cada entorno}
\label{tab:sufijos}
\begin{tabular}{lll}
\toprule
\textbf{Sufijo} & \textbf{Descripción} & \textbf{Ejemplo} \\
\midrule
(ninguno) & Light con números & \texttt{pythoncode} \\
\texttt{NN} & Light sin números & \texttt{pythoncodeNN} \\
\texttt{Dark} & Dark con números & \texttt{pythoncodeDark} \\
\texttt{DarkNN} & Dark sin números & \texttt{pythoncodeDarkNN} \\
\bottomrule
\end{tabular}
\end{table}

\begin{table}[H]
\centering
\caption{Entornos genéricos para cualquier lenguaje}
\label{tab:generico}
\begin{tabular}{ll}
\toprule
\textbf{Entorno} & \textbf{Descripción} \\
\midrule
\texttt{codigo\{lang\}} & Light con números \\
\texttt{codigoNN\{lang\}} & Light sin números \\
\texttt{codigoDark\{lang\}} & Dark con números \\
\texttt{codigoDarkNN\{lang\}} & Dark sin números \\
\bottomrule
\end{tabular}
\end{table}
