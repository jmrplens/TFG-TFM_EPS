\chapter{Metodología}
\label{ch:metodologia}

Este capítulo describe la metodología seguida para el desarrollo del trabajo.

\section{Metodología de desarrollo}

Para el desarrollo de este trabajo se ha seguido una metodología ágil basada en...

\subsection{Fases del proyecto}

El proyecto se ha dividido en las siguientes fases:

\begin{enumerate}
  \item \textbf{Fase de análisis:} Estudio del problema y definición de requisitos.
  \item \textbf{Fase de diseño:} Diseño de la arquitectura y los componentes.
  \item \textbf{Fase de implementación:} Desarrollo del código y funcionalidades.
  \item \textbf{Fase de pruebas:} Verificación y validación del sistema.
  \item \textbf{Fase de documentación:} Elaboración de la memoria y manuales.
\end{enumerate}

\section{Planificación temporal}

La planificación temporal del proyecto se muestra en la Figura~\ref{fig:gantt}.

\begin{figure}[htbp]
  \centering
  \begin{tikzpicture}[
    fase/.style={fill=informatica!60, draw=informatica!80, rounded corners=2pt, minimum height=0.6cm},
    mes/.style={font=\footnotesize\sffamily},
    nombre/.style={font=\footnotesize, anchor=east}
  ]
    % Eje temporal (meses)
    \foreach \m/\n in {1/Sep, 2/Oct, 3/Nov, 4/Dic, 5/Ene, 6/Feb} {
      \node[mes] at (\m*1.5, 4.5) {\n};
    }
    
    % Líneas de guía verticales
    \foreach \m in {1,...,6} {
      \draw[gray!30] (\m*1.5-0.75, 0.5) -- (\m*1.5-0.75, 4);
    }
    
    % Fases del proyecto
    \node[nombre] at (0, 3.5) {Análisis};
    \draw[fase] (0.75, 3.3) rectangle (2.25, 3.7);
    
    \node[nombre] at (0, 2.8) {Diseño};
    \draw[fase] (1.5, 2.6) rectangle (3.75, 3);
    
    \node[nombre] at (0, 2.1) {Implementación};
    \draw[fase] (3, 1.9) rectangle (6.75, 2.3);
    
    \node[nombre] at (0, 1.4) {Pruebas};
    \draw[fase] (5.25, 1.2) rectangle (7.5, 1.6);
    
    \node[nombre] at (0, 0.7) {Documentación};
    \draw[fase] (6, 0.5) rectangle (9, 0.9);
  \end{tikzpicture}
  \caption{Planificación temporal del proyecto (diagrama de Gantt simplificado)}
  \label{fig:gantt}
\end{figure}

\section{Recursos utilizados}

\subsection{Recursos hardware}

\begin{itemize}
  \item Ordenador portátil con procesador Intel i7, 16GB RAM
  \item Servidor de desarrollo con 32GB RAM
\end{itemize}

\subsection{Recursos software}

\begin{itemize}
  \item Sistema operativo: Linux Ubuntu 24.04 LTS
  \item Entorno de desarrollo: Visual Studio Code
  \item Control de versiones: Git y GitHub
\end{itemize}

\section{Gestión del proyecto}

Para la gestión del proyecto se han utilizado las siguientes herramientas:

\begin{itemize}
  \item \textbf{GitHub Projects:} Para la gestión de tareas y seguimiento del progreso.
  \item \textbf{Git:} Para el control de versiones del código.
  \item \textbf{Discord:} Para la comunicación con el tutor.
\end{itemize}
