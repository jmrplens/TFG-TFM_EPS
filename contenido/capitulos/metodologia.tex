\chapter{Metodología}
\label{ch:metodologia}

Este capítulo describe la metodología seguida para el desarrollo del trabajo.

\section{Metodología de desarrollo}

Para el desarrollo de este trabajo se ha seguido una metodología ágil basada en...

\subsection{Fases del proyecto}

El proyecto se ha dividido en las siguientes fases:

\begin{enumerate}
  \item \textbf{Fase de análisis:} Estudio del problema y definición de requisitos.
  \item \textbf{Fase de diseño:} Diseño de la arquitectura y los componentes.
  \item \textbf{Fase de implementación:} Desarrollo del código y funcionalidades.
  \item \textbf{Fase de pruebas:} Verificación y validación del sistema.
  \item \textbf{Fase de documentación:} Elaboración de la memoria y manuales.
\end{enumerate}

\section{Planificación temporal}

La planificación temporal del proyecto se muestra en la Figura~\ref{fig:gantt}.

\begin{figure}[htbp]
  \centering
  \missingfigure{Diagrama de Gantt con la planificación del proyecto}
  \caption{Planificación temporal del proyecto}
  \label{fig:gantt}
\end{figure}

\section{Recursos utilizados}

\subsection{Recursos hardware}

\begin{itemize}
  \item Ordenador portátil con procesador Intel i7, 16GB RAM
  \item Servidor de desarrollo con 32GB RAM
\end{itemize}

\subsection{Recursos software}

\begin{itemize}
  \item Sistema operativo: Linux Ubuntu 24.04 LTS
  \item Entorno de desarrollo: Visual Studio Code
  \item Control de versiones: Git y GitHub
\end{itemize}

\section{Gestión del proyecto}

Para la gestión del proyecto se han utilizado las siguientes herramientas:

\begin{itemize}
  \item \textbf{GitHub Projects:} Para la gestión de tareas y seguimiento del progreso.
  \item \textbf{Git:} Para el control de versiones del código.
  \item \textbf{Discord:} Para la comunicación con el tutor.
\end{itemize}
