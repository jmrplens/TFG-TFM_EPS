\chapter{Marco Teórico (Ejemplos de listas)}
\label{ch:marco-teorico}

Este capítulo presenta los fundamentos teóricos necesarios y demuestra las diferentes formas de crear listas en \LaTeX.

\section{Listas básicas}

Existen tres tipos principales de listas en \LaTeX: itemize (viñetas), enumerate (numeradas) y description (definiciones).

\subsection{Listas con viñetas (itemize)}

\begin{latexcode}[title={Lista con viñetas}]
\begin{itemize}
  \item Primer elemento
  \item Segundo elemento
  \item Tercer elemento
\end{itemize}
\end{latexcode}

El resultado es:

\begin{itemize}
  \item Primer elemento de la lista
  \item Segundo elemento de la lista
  \item Tercer elemento de la lista
\end{itemize}

\subsection{Listas numeradas (enumerate)}

\begin{latexcode}[title={Lista numerada}]
\begin{enumerate}
  \item Primer paso del proceso
  \item Segundo paso del proceso
  \item Tercer paso del proceso
\end{enumerate}
\end{latexcode}

El resultado es:

\begin{enumerate}
  \item Primer paso del proceso
  \item Segundo paso del proceso
  \item Tercer paso del proceso
\end{enumerate}

\subsection{Listas anidadas}

Las listas pueden anidarse hasta varios niveles:

\begin{itemize}
  \item Ingeniería Informática
    \begin{itemize}
      \item Mención en Computación
      \item Mención en Ingeniería del Software
      \item Mención en Sistemas de Información
    \end{itemize}
  \item Ingeniería Multimedia
    \begin{itemize}
      \item Mención en Creación y Ocio Digital
      \item Mención en Gestión de Contenidos
    \end{itemize}
  \item Ingeniería en Sonido e Imagen
\end{itemize}

También con números:

\begin{enumerate}
  \item Fase de análisis
    \begin{enumerate}
      \item Recopilación de requisitos
      \item Análisis de viabilidad
      \item Documentación inicial
    \end{enumerate}
  \item Fase de diseño
    \begin{enumerate}
      \item Diseño arquitectónico
      \item Diseño detallado
    \end{enumerate}
  \item Fase de implementación
\end{enumerate}

\section{Listas de definición}

Las listas de descripción son útiles para glosarios y definiciones:

\begin{latexcode}[title={Lista de definiciones}]
\begin{description}
  \item[Término 1:] Definición del primer término.
  \item[Término 2:] Definición del segundo término.
\end{description}
\end{latexcode}

\begin{description}
  \item[LaTeX:] Sistema de composición de textos orientado a la creación de documentos científicos y técnicos de alta calidad tipográfica.
  
  \item[TFG:] Trabajo Fin de Grado. Proyecto final que los estudiantes de grado deben realizar y defender para obtener su título universitario.
  
  \item[TFM:] Trabajo Fin de Máster. Proyecto de investigación o aplicación que los estudiantes de máster deben completar.
  
  \item[EPS:] Escuela Politécnica Superior de la Universidad de Alicante.
\end{description}

\subsection{Listas de definición anidadas}

\begin{description}
  \item[Software de diseño:] Herramientas utilizadas en el proyecto.
    \begin{description}
      \item[Ventajas:]~
        \begin{itemize}
          \item Permite realizar simulaciones precisas
          \item Interfaz gráfica intuitiva
          \item Exportación a múltiples formatos
        \end{itemize}
      \item[Inconvenientes:]~
        \begin{itemize}
          \item Curva de aprendizaje pronunciada
          \item Requiere licencia comercial
          \item Alto consumo de recursos
        \end{itemize}
    \end{description}
\end{description}

\section{Estado del arte}

A continuación se presenta una revisión de los trabajos más relevantes en el área.

\subsection{Trabajos previos}

Según \textcite{latex2024}, el estado actual de la investigación en sistemas de composición tipográfica indica que \LaTeX{} sigue siendo el estándar para documentos científicos.

En \cite{overleaf2024} se propone un enfoque basado en edición colaborativa en la nube que ha facilitado enormemente la adopción de \LaTeX{} en entornos académicos.

\subsection{Tecnologías relacionadas}

Las principales tecnologías utilizadas en este ámbito incluyen:

\begin{enumerate}
  \item \textbf{LuaLaTeX:} Motor de composición moderno que permite:
    \begin{itemize}
      \item Uso directo de fuentes del sistema
      \item Programación en Lua dentro del documento
      \item Soporte nativo de Unicode
    \end{itemize}
    
  \item \textbf{Overleaf:} Plataforma de edición online que ofrece:
    \begin{itemize}
      \item Compilación en la nube
      \item Colaboración en tiempo real
      \item Control de versiones integrado
    \end{itemize}
    
  \item \textbf{Git:} Sistema de control de versiones útil para:
    \begin{itemize}
      \item Seguimiento de cambios en el documento
      \item Trabajo en equipo
      \item Recuperación de versiones anteriores
    \end{itemize}
\end{enumerate}

\section{Herramientas utilizadas}

En este trabajo se han utilizado las siguientes herramientas:

\begin{table}[H]
  \centering
  \caption{Herramientas utilizadas en el desarrollo}
  \label{tab:herramientas}
  \begin{tabular}{@{}llr@{}}
    \toprule
    \textbf{Herramienta} & \textbf{Versión} & \textbf{Uso} \\
    \midrule
    TeX Live & 2024 & Distribución LaTeX \\
    LuaLaTeX & 1.18 & Motor de compilación \\
    VS Code & 1.85 & Editor de texto \\
    Git & 2.43 & Control de versiones \\
    Pygments & 2.17 & Resaltado de código \\
    \bottomrule
  \end{tabular}
\end{table}

La Tabla~\ref{tab:herramientas} muestra las principales herramientas empleadas en la elaboración de este documento.
