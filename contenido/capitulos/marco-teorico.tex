\chapter{Marco teórico}
\label{ch:marco-teorico}

Este capítulo presenta los fundamentos teóricos necesarios para el desarrollo del trabajo.

\section{Fundamentos básicos}

En esta sección se describen los conceptos fundamentales relacionados con el trabajo.

\subsection{Concepto A}

Lorem ipsum dolor sit amet, consectetur adipiscing elit. Sed do eiusmod tempor incididunt ut labore et dolore magna aliqua.

\subsection{Concepto B}

Ut enim ad minim veniam, quis nostrud exercitation ullamco laboris nisi ut aliquip ex ea commodo consequat.

\section{Estado del arte}

A continuación se presenta una revisión de los trabajos más relevantes en el área.

\subsection{Trabajos previos}

Según \textcite{ejemplo2024}, el estado actual de la investigación indica que...

En \cite{ejemplo2023} se propone un enfoque alternativo basado en...

\subsection{Tecnologías relacionadas}

Las principales tecnologías utilizadas en este ámbito incluyen:

\begin{itemize}
  \item \textbf{Tecnología 1:} Descripción breve de la primera tecnología.
  \item \textbf{Tecnología 2:} Descripción de la segunda tecnología.
  \item \textbf{Tecnología 3:} Características de la tercera tecnología.
\end{itemize}

\section{Herramientas y tecnologías utilizadas}

En este trabajo se han utilizado las siguientes herramientas:

\begin{table}[htbp]
  \centering
  \caption{Herramientas utilizadas en el desarrollo}
  \label{tab:herramientas}
  \begin{tabular}{@{}llr@{}}
    \toprule
    \textbf{Herramienta} & \textbf{Versión} & \textbf{Uso} \\
    \midrule
    Python & 3.12 & Desarrollo principal \\
    PostgreSQL & 16.0 & Base de datos \\
    Docker & 25.0 & Contenedorización \\
    Git & 2.43 & Control de versiones \\
    \bottomrule
  \end{tabular}
\end{table}

La Tabla~\ref{tab:herramientas} muestra las principales herramientas empleadas.
