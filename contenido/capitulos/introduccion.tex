\chapter{Introducción (Ejemplos de contenido y estilos)}
\label{ch:introduccion}

Este capítulo presenta una guía completa de las capacidades de \LaTeX{} y de esta
plantilla para la elaboración de Trabajos Fin de Grado y Trabajos Fin de Máster
en la Escuela Politécnica Superior de la Universidad de Alicante. Incluye ejemplos
de todos los elementos básicos de formato y estructura que necesitarás para
redactar tu trabajo académico.

%% =============================================================================
\section{Sobre esta plantilla}
\label{sec:sobre-plantilla}

Esta plantilla ha sido diseñada siguiendo las directrices de estilo de la \gls{eps} de la Universidad de Alicante. Proporciona una estructura clara y profesional para la redacción de trabajos académicos.

Las principales características de esta plantilla son:

\begin{itemize}
  \item Configuración sencilla mediante clave-valor con \verb|\EPSsetup{...}|
  \item 21 portadas diferentes según la titulación
  \item Soporte para múltiples idiomas (español, inglés, valenciano)
  \item Resaltado de código con colores (minted + Pygments)
  \item Bibliografía con estilo \gls{apa}
  \item Compilación con LuaLaTeX
\end{itemize}

%% =============================================================================
\section{Estructura de un TFG/TFM}
\label{sec:estructura-tfg}

Según las normas de la \gls{eps}, un \gls{tfg} o \gls{tfm} debe contener
las siguientes partes obligatorias y opcionales:

\begin{description}
  \item[Preámbulo:] Motivación y descripción breve de los objetivos del trabajo.
  \item[Agradecimientos:] Reconocimientos a entidades y personas colaboradoras.
  \item[Dedicatoria:] Opcional, con alineación a la derecha.
  \item[Índices:] Contenidos, figuras, tablas y códigos.
  \item[Introducción:] Importancia de la temática, vigencia y planteamiento del problema.
  \item[Marco teórico:] Fundamentos conceptuales y estado del arte.
  \item[Objetivos:] Objetivo general y específicos.
  \item[Metodología:] Tipo de investigación, técnicas y procedimientos.
  \item[Resultados:] Resultados obtenidos y análisis.
  \item[Conclusiones:] Resumen de objetivos conseguidos.
  \item[Bibliografía:] Referencias utilizadas (estilo APA recomendado).
  \item[Anexos:] Material complementario.
\end{description}

%% =============================================================================
\section{Secciones y subsecciones}
\label{sec:secciones}

En \LaTeX{} existen diferentes niveles de organización del contenido.
Cada nivel tiene una numeración automática y aparece en el índice según
su jerarquía.

\begin{verbatim}
\chapter{Capítulo}           % Nivel 0
\section{Sección}            % Nivel 1
\subsection{Subsección}      % Nivel 2
\subsubsection{Subsubsección}% Nivel 3
\paragraph{Párrafo}          % Nivel 4
\end{verbatim}

\subsection{Ejemplo de subsección}

Este es el contenido de una subsección. Las subsecciones permiten organizar mejor el contenido dentro de cada sección.

\subsubsection{Ejemplo de subsubsección}

Las subsubsecciones son útiles para temas muy específicos dentro de una subsección.

\paragraph{Ejemplo de párrafo} Los párrafos con título son útiles para pequeñas divisiones que no necesitan aparecer en el índice.

%% =============================================================================
\section{Citas y bibliografía}
\label{sec:citas}

La plantilla utiliza \textbf{BibLaTeX} con el backend \textbf{Biber} y estilo
\textbf{APA 7} para la gestión de referencias bibliográficas. Las referencias
se almacenan en el archivo \texttt{referencias.bib}.

\subsection{Comandos de citación}
\label{subsec:comandos-citacion}

BibLaTeX ofrece varios comandos para citar según el contexto:

\begin{verbatim}
% Cita textual: Autor (año)
\textcite{clave}

% Cita entre paréntesis: (Autor, año)
\parencite{clave}

% Solo autor: Autor
\citeauthor{clave}

% Solo año: (año)
\citeyear{clave}

% Cita con página específica
\parencite[p.~25]{clave}

% Múltiples citas
\parencite{clave1,clave2,clave3}

% Cita completa en nota al pie
\footcite{clave}
\end{verbatim}

Ejemplos de uso:
\begin{itemize}
  \item Cita textual: Según \textcite{latex2024}, \LaTeX{} es el estándar para documentos científicos.
  \item Cita entre paréntesis: \LaTeX{} es ampliamente utilizado \parencite{latex2024}.
  \item Múltiples fuentes: Existen varios recursos disponibles \parencite{latex2024,overleaf2024}.
\end{itemize}

\subsection{Tipos de entradas bibliográficas}
\label{subsec:tipos-entradas-bib}

El archivo \texttt{.bib} admite diferentes tipos de fuentes. A continuación
se muestran los más comunes con su estructura:

\subsubsection{Libro (@book)}

\begin{verbatim}
@book{knuth1984texbook,
  author    = {Knuth, Donald E.},
  title     = {The TeXbook},
  publisher = {Addison-Wesley},
  year      = {1984},
  address   = {Reading, MA},
  isbn      = {0-201-13447-0}
}
\end{verbatim}

\subsubsection{Artículo de revista (@article)}

\begin{verbatim}
@article{shannon1948mathematical,
  author  = {Shannon, Claude E.},
  title   = {A Mathematical Theory of Communication},
  journal = {Bell System Technical Journal},
  year    = {1948},
  volume  = {27},
  number  = {3},
  pages   = {379--423},
  doi     = {10.1002/j.1538-7305.1948.tb01338.x}
}
\end{verbatim}

\subsubsection{Conferencia (@inproceedings)}

\begin{verbatim}
@inproceedings{krizhevsky2012imagenet,
  author    = {Krizhevsky, Alex and Sutskever, Ilya
               and Hinton, Geoffrey E.},
  title     = {ImageNet Classification with Deep
               Convolutional Neural Networks},
  booktitle = {Advances in Neural Information
               Processing Systems},
  year      = {2012},
  pages     = {1097--1105},
  publisher = {Curran Associates}
}
\end{verbatim}

\subsubsection{Recurso en línea (@online)}

\begin{verbatim}
@online{overleaf2024,
  author  = {{Overleaf}},
  title   = {Documentation},
  url     = {https://www.overleaf.com/learn},
  urldate = {2024-01-15},
  year    = {2024}
}
\end{verbatim}

\subsubsection{Tesis (@thesis)}

\begin{verbatim}
@thesis{garcia2023sistema,
  author      = {García López, María},
  title       = {Sistema de gestión inteligente},
  type        = {Trabajo Fin de Grado},
  institution = {Universidad de Alicante},
  year        = {2023}
}
\end{verbatim}

\subsubsection{Manual o documentación técnica (@manual)}

\begin{verbatim}
@manual{latex2024,
  author       = {{LaTeX Project}},
  title        = {LaTeX2e: An unofficial reference manual},
  year         = {2024},
  organization = {LaTeX Project},
  url          = {https://latexref.xyz/}
}
\end{verbatim}

\subsection{Campos comunes en BibLaTeX}
\label{subsec:campos-biblatex}

\begin{table}[H]
  \centering
  \caption{Campos más utilizados en entradas bibliográficas}
  \label{tab:campos-bib}
  \begin{tabular}{lp{8cm}}
    \toprule
    \textbf{Campo} & \textbf{Descripción} \\
    \midrule
    \texttt{author}    & Autor(es), separados por \texttt{and} \\
    \texttt{title}     & Título de la obra \\
    \texttt{year}      & Año de publicación \\
    \texttt{publisher} & Editorial \\
    \texttt{journal}   & Nombre de la revista (para @article) \\
    \texttt{booktitle} & Título del libro/actas (para @inproceedings) \\
    \texttt{volume}    & Volumen de la revista \\
    \texttt{number}    & Número de la revista \\
    \texttt{pages}     & Páginas (usar \texttt{--} para rango) \\
    \texttt{doi}       & Digital Object Identifier \\
    \texttt{url}       & Enlace web \\
    \texttt{urldate}   & Fecha de acceso (formato YYYY-MM-DD) \\
    \texttt{isbn}      & ISBN del libro \\
    \texttt{issn}      & ISSN de la revista \\
    \bottomrule
  \end{tabular}
\end{table}

\begin{tipbox}{Buenas prácticas bibliográficas}
  \begin{itemize}[noitemsep]
    \item Usa claves descriptivas: \texttt{apellidoAñoPalabra} (ej: \texttt{knuth1984texbook})
    \item Incluye siempre el DOI cuando esté disponible
    \item Para recursos web, añade siempre \texttt{urldate}
    \item Usa llaves \texttt{\{\}} para preservar mayúsculas en títulos: \texttt{\{DNA\}}
    \item Para instituciones como autor, usa dobles llaves: \texttt{\{\{OMS\}\}}
  \end{itemize}
\end{tipbox}

\subsection{Compilación con Biber}
\label{subsec:compilacion-biber}

Para procesar la bibliografía correctamente, se requiere ejecutar Biber
entre compilaciones de \LaTeX{}:

\begin{verbatim}
lualatex main.tex    % Primera pasada
biber main           % Procesa bibliografía
lualatex main.tex    % Segunda pasada (resuelve citas)
lualatex main.tex    % Tercera pasada (actualiza índices)
\end{verbatim}

\begin{infobox}{Usando latexmk}
  El comando \texttt{latexmk -lualatex -shell-escape main.tex} ejecuta
  automáticamente todas las pasadas necesarias, incluyendo Biber.
\end{infobox}

%% =============================================================================
\section{Notas al pie de página}
\label{sec:notas-pie}

Las notas al pie permiten añadir información adicional sin interrumpir
el flujo del texto principal. Son útiles para aclaraciones, definiciones
o referencias complementarias.

\begin{verbatim}
Texto principal\footnote{Contenido de la nota al pie.}
\end{verbatim}

La plantilla utiliza LuaLaTeX\footnote{LuaLaTeX es un motor de composición tipográfica que combina \LaTeX{} con el lenguaje de programación Lua, permitiendo mayor flexibilidad en el procesamiento de documentos.} como motor de compilación por sus capacidades avanzadas de manejo de fuentes\footnote{Permite usar cualquier fuente OpenType o TrueType instalada en el sistema sin necesidad de configuración adicional.}.

%% =============================================================================
\section{Estilos de texto}
\label{sec:estilos-texto}

\LaTeX{} ofrece múltiples estilos de texto para dar énfasis o formato
especial a determinadas palabras o frases.

\begin{verbatim}
\textit{Texto en cursiva}
\textbf{Texto en negrita}
\texttt{Texto monoespacio}
\textsc{Texto en versalitas}
\underline{Texto subrayado}
\textbf{\textit{Combinación de estilos}}
{\small Texto pequeño}
{\large Texto grande}
\end{verbatim}

Resultado de cada estilo:

\begin{itemize}
  \item \textit{Texto en cursiva} --- \verb|\textit{texto}|
  \item \textbf{Texto en negrita} --- \verb|\textbf{texto}|
  \item \texttt{Texto monoespacio} --- \verb|\texttt{texto}|
  \item \textsc{Texto en versalitas} --- \verb|\textsc{texto}|
  \item \underline{Texto subrayado} --- \verb|\underline{texto}|
  \item \textbf{\textit{Negrita y cursiva}} --- combinación de comandos
  \item {\small Texto pequeño} --- \verb|{\small texto}|
  \item {\large Texto grande} --- \verb|{\large texto}|
\end{itemize}

%% =============================================================================
\section{Acrónimos y glosario}
\label{sec:acronimos}

La plantilla gestiona automáticamente acrónimos y términos del glosario con el
paquete \texttt{glossaries}. Ambos se definen en \texttt{contenido/anexos/acronimos.tex}.

\subsection{Uso de acrónimos}
\label{subsec:uso-acronimos}

La primera vez que aparece un acrónimo se muestra su forma completa, y en
las siguientes apariciones solo la abreviatura:

\begin{verbatim}
% Primera aparición: "Institute of Electrical
% and Electronics Engineers (IEEE)"
El \gls{ieee} es una institución importante.

% Siguientes apariciones: muestra solo "IEEE"
El \gls{ieee} establece estándares.

% Forzar forma larga o corta
\acrlong{ieee}   % Institute of Electrical...
\acrshort{ieee}  % IEEE
\end{verbatim}

Ejemplo: El \gls{ieee} es una institución importante. El \gls{ieee} establece estándares. Otras organizaciones incluyen la \gls{iso} y el \gls{w3c}.

\subsection{Uso del glosario de términos}
\label{subsec:uso-glosario}

Los términos del glosario funcionan igual que los acrónimos pero para
conceptos que requieren definición:

\begin{verbatim}
% Usar un término (singular)
Un \gls{algoritmo} es la base de la programación.

% Usar un término (plural)
Los \glspl{framework} facilitan el desarrollo.

% Forzar primera letra mayúscula
\Gls{backend} y \gls{frontend} trabajan juntos.
\end{verbatim}

Ejemplo: Un \gls{algoritmo} bien diseñado mejora el rendimiento. Los \glspl{microservicio} permiten \gls{escalabilidad}. El \gls{backend} procesa las peticiones del \gls{frontend}.

\begin{tipbox}{Definir nuevos términos}
  \begin{verbatim}
\newglossaryentry{etiqueta}{
  name={término visible},
  plural={términos visibles},
  description={Definición del término}
}
  \end{verbatim}
\end{tipbox}

%% =============================================================================
\section{Tareas pendientes y notas}
\label{sec:todonotes}

Durante la redacción es útil marcar partes que requieren revisión.
El paquete \texttt{todonotes} permite añadir notas visuales que son fáciles
de localizar y eliminar antes de la entrega final.

\begin{verbatim}
% Nota al margen
\todo{Texto de la nota}

% Nota en línea
\todo[inline]{Nota dentro del texto}

% Figura pendiente
\missingfigure{Descripción de la figura}
\end{verbatim}

\todo{Esta es una nota al margen para recordar algo}

Las notas también pueden incluirse dentro del texto:

\todo[inline]{Esta es una nota en línea que indica trabajo pendiente para esta sección. Muy útil durante la fase de redacción.}

Para indicar figuras que faltan por añadir:

\missingfigure{Aquí irá un diagrama explicativo del flujo de trabajo}

%% =============================================================================
\section{Comandos personalizados de la plantilla}
\label{sec:comandos-eps}

Esta plantilla proporciona comandos para acceder a la información configurada
en \texttt{configuracion.tex}. Son útiles para incluir datos del trabajo
dinámicamente en el documento.

\begin{verbatim}
\EPStitulo      % Título del trabajo
\EPSautor       % Nombre del autor
\EPStutor       % Nombre del tutor
\EPSfecha       % Fecha de presentación
\EPStipoTrabajo % TFG o TFM
\EPStitulacion  % Nombre de la titulación
\end{verbatim}

Resultado de cada comando:

\begin{itemize}
  \item \verb|\EPStitulo| --- ``\EPStitulo''
  \item \verb|\EPSautor| --- \EPSautor
  \item \verb|\EPStutor| --- \EPStutor
  \item \verb|\EPSfecha| --- \EPSfecha
  \item \verb|\EPStipoTrabajo| --- \EPStipoTrabajo
  \item \verb|\EPStitulacion| --- \EPStitulacion
\end{itemize}

Ejemplo de uso en el texto: Este documento es un \EPStipoTrabajo{} presentado en \EPSfecha.

%% =============================================================================
\section{Hipervínculos y URLs}
\label{sec:hipervinculos}

Para incluir enlaces web se utilizan comandos del paquete \texttt{hyperref},
que ya viene configurado en esta plantilla.

\begin{verbatim}
% URL con formato automático (monoespacio)
\url{https://eps.ua.es}

% Enlace con texto personalizado
\href{https://eps.ua.es}{Escuela Politécnica Superior}

% Referencia a email
\href{mailto:autor@ejemplo.com}{autor@ejemplo.com}
\end{verbatim}

Ejemplos:

\begin{itemize}
  \item URL simple: \url{https://eps.ua.es}
  \item Overleaf (editor online): \url{https://www.overleaf.com}
  \item CTAN (repositorio de paquetes): \url{https://ctan.org}
  \item Enlace con texto: \href{https://eps.ua.es}{Escuela Politécnica Superior}
\end{itemize}

%% =============================================================================
\section{Listas}
\label{sec:listas}

\LaTeX{} ofrece tres tipos principales de listas para organizar información.
Cada tipo tiene un propósito específico y un formato visual diferente.

\subsection{Lista sin numerar (itemize)}

Para elementos sin orden específico.

\begin{verbatim}
\begin{itemize}
  \item Primer elemento
  \item Segundo elemento
  \item Tercer elemento
\end{itemize}
\end{verbatim}

\begin{itemize}
  \item Primer elemento de la lista
  \item Segundo elemento de la lista
  \item Tercer elemento de la lista
\end{itemize}

\subsection{Lista numerada (enumerate)}

Para elementos con orden secuencial.

\begin{verbatim}
\begin{enumerate}
  \item Paso uno
  \item Paso dos
  \item Paso tres
\end{enumerate}
\end{verbatim}

\begin{enumerate}
  \item Preparar el entorno de desarrollo
  \item Implementar la funcionalidad principal
  \item Realizar pruebas y validación
\end{enumerate}

\subsection{Lista de descripción (description)}

Para términos con sus definiciones.

\begin{verbatim}
\begin{description}
  \item[Término 1] Definición del primer término.
  \item[Término 2] Definición del segundo término.
\end{description}
\end{verbatim}

\begin{description}
  \item[LuaLaTeX] Motor de compilación que combina \LaTeX{} con Lua.
  \item[BibLaTeX] Sistema moderno de gestión de bibliografía.
  \item[TikZ] Paquete para crear gráficos vectoriales programáticamente.
\end{description}

\subsection{Listas anidadas}

Las listas pueden anidarse hasta tres niveles de profundidad.

\begin{verbatim}
\begin{itemize}
  \item Elemento principal
  \begin{itemize}
    \item Subelemento
    \begin{itemize}
      \item Sub-subelemento
    \end{itemize}
  \end{itemize}
\end{itemize}
\end{verbatim}

\begin{itemize}
  \item Requisitos funcionales
  \begin{itemize}
    \item Gestión de usuarios
    \begin{itemize}
      \item Registro
      \item Autenticación
      \item Recuperación de contraseña
    \end{itemize}
    \item Gestión de contenido
  \end{itemize}
  \item Requisitos no funcionales
  \begin{itemize}
    \item Rendimiento
    \item Seguridad
  \end{itemize}
\end{itemize}

%% =============================================================================
\section{Referencias cruzadas}
\label{sec:referencias-cruzadas}

Las referencias cruzadas permiten enlazar elementos del documento
(capítulos, secciones, figuras, tablas, ecuaciones) de forma que
los números se actualicen automáticamente si cambia la estructura.

\begin{verbatim}
% Crear etiqueta (después de \chapter, \section, \caption, etc.)
\label{tipo:nombre}

% Referenciar la etiqueta
En la Sección~\ref{sec:listas} se explican las listas.
En la Figura~\ref{fig:ejemplo} se muestra...
En la Tabla~\ref{tab:datos} aparecen...
En la página~\pageref{sec:listas} comienza...
\end{verbatim}

\minisec{Prefijos recomendados para etiquetas}

Para mantener el código organizado, usa prefijos consistentes:

\begin{description}
  \item[ch:] Capítulos --- \verb|\label{ch:introduccion}|
  \item[sec:] Secciones --- \verb|\label{sec:metodologia}|
  \item[fig:] Figuras --- \verb|\label{fig:diagrama}|
  \item[tab:] Tablas --- \verb|\label{tab:resultados}|
  \item[eq:] Ecuaciones --- \verb|\label{eq:formula}|
  \item[lst:] Códigos --- \verb|\label{lst:algoritmo}|
  \item[apx:] Anexos --- \verb|\label{apx:manual}|
\end{description}

\minisec{Ejemplos de uso}

\begin{itemize}
  \item Referencia a sección: Ver la Sección~\ref{sec:listas} para información sobre listas.
  \item Referencia a página: Las citas se explican en la página~\pageref{sec:citas}.
  \item Referencia a capítulo: El Capítulo~\ref{ch:introduccion} contiene ejemplos básicos.
\end{itemize}

\begin{tipbox}
  Usa siempre \verb|~| (espacio insecable) entre el tipo de elemento y
  \verb|\ref{}| para evitar que el número quede separado del texto en
  un salto de línea: \verb|Figura~\ref{fig:ejemplo}|.
\end{tipbox}

%% =============================================================================
\section{Consejos para la redacción}
\label{sec:consejos}

A continuación se presentan algunas recomendaciones para la redacción del \gls{tfg} o \gls{tfm}:

\subsection{Estructura del texto}

\begin{itemize}
  \item Utiliza párrafos cortos y concisos (máximo 8-10 líneas)
  \item Comienza cada capítulo con una breve introducción de su contenido
  \item Finaliza cada capítulo con un resumen o transición al siguiente
  \item Evita frases demasiado largas; divide las ideas complejas
\end{itemize}

\subsection{Estilo académico}

\begin{itemize}
  \item Escribe en tercera persona o forma impersonal (<<se ha desarrollado>> en lugar de <<he desarrollado>>)
  \item Mantén un tono formal y objetivo
  \item Evita expresiones coloquiales o jerga no técnica
  \item Define los términos técnicos la primera vez que aparecen
\end{itemize}

\subsection{Referencias y citas}

\begin{itemize}
  \item Cita siempre las fuentes de información que no sean tuyas
  \item Usa \verb|\parencite{clave}| para citas entre paréntesis: \parencite{knuth1984texbook}
  \item Usa \verb|\textcite{clave}| para citas narrativas: \textcite{lamport1994latex}
  \item Evita el exceso de citas; selecciona las más relevantes
  \item Verifica que todas las referencias del archivo \texttt{.bib} estén citadas
\end{itemize}

\subsection{Figuras y tablas}

\begin{itemize}
  \item Toda figura y tabla debe estar referenciada en el texto
  \item Las descripciones (captions) deben ser autoexplicativas
  \item Numera las figuras y tablas de forma consecutiva
  \item Usa formatos vectoriales (PDF, SVG) cuando sea posible
  \item Mantén consistencia en el estilo visual
\end{itemize}

\subsection{Revisión final}

Antes de entregar, verifica:

\begin{enumerate}
  \item Ortografía y gramática (usa correctores automáticos)
  \item Referencias cruzadas funcionan correctamente
  \item Bibliografía está completa y correctamente formateada
  \item Índices están actualizados
  \item Márgenes y formato cumplen la normativa
  \item No hay notas \texttt{\textbackslash todo} pendientes
\end{enumerate}
