\chapter{Introducción}
\label{ch:introduccion}

Este capítulo presenta el contexto, la motivación y la estructura del trabajo.

\section{Contexto y motivación}

El presente \EPStipoTrabajo{} se desarrolla en el marco de la titulación de \EPStitulacion{} de la \EPSfacultad{} de la \EPSuniversidad{}.

La motivación principal de este trabajo surge de la necesidad de... \todo{Completar la motivación del trabajo}

\section{Planteamiento del problema}

El problema que se pretende abordar en este trabajo consiste en...

Se pueden identificar los siguientes aspectos clave:

\begin{enumerate}
  \item Primer aspecto del problema
  \item Segundo aspecto a considerar
  \item Tercer elemento relevante
\end{enumerate}

\section{Estructura del documento}

El presente documento se organiza de la siguiente manera:

\begin{description}
  \item[Capítulo \ref{ch:marco-teorico}: Marco teórico] Se presentan los fundamentos teóricos necesarios para comprender el trabajo.
  
  \item[Capítulo \ref{ch:objetivos}: Objetivos] Se definen los objetivos generales y específicos del trabajo.
  
  \item[Capítulo \ref{ch:metodologia}: Metodología] Se describe la metodología empleada para el desarrollo del trabajo.
  
  \item[Capítulo \ref{ch:desarrollo}: Desarrollo] Se detalla el desarrollo técnico del trabajo.
  
  \item[Capítulo \ref{ch:resultados}: Resultados] Se presentan y analizan los resultados obtenidos.
  
  \item[Capítulo \ref{ch:conclusiones}: Conclusiones] Se exponen las conclusiones y posibles líneas de trabajo futuro.
\end{description}
