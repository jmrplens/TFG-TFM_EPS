\chapter{Introducción (Ejemplos de contenido y estilos)}
\label{ch:introduccion}

Este capítulo presenta una guía completa de las capacidades de \LaTeX{} y de esta plantilla para la elaboración de Trabajos Fin de Grado y Trabajos Fin de Máster en la Escuela Politécnica Superior de la Universidad de Alicante.

\section{Sobre esta plantilla}

Esta plantilla ha sido diseñada siguiendo las directrices de estilo de la \gls{eps} de la Universidad de Alicante. Proporciona una estructura clara y profesional para la redacción de trabajos académicos.

Las principales características de esta plantilla son:

\begin{itemize}
  \item Configuración sencilla mediante clave-valor con \verb|\EPSsetup{...}|
  \item 21 portadas diferentes según la titulación
  \item Soporte para múltiples idiomas (español, inglés, valenciano)
  \item Resaltado de código con colores (minted + Pygments)
  \item Bibliografía con estilo \gls{apa}
  \item Compilación con LuaLaTeX
\end{itemize}

\section{Estructura de un TFG/TFM}

Según las normas de la \gls{eps}, un \gls{tfg} o \gls{tfm} debe contener las siguientes partes:

\begin{description}
  \item[Preámbulo:] Motivación y descripción breve de los objetivos del trabajo.
  \item[Agradecimientos:] Reconocimientos a entidades y personas colaboradoras.
  \item[Dedicatoria:] Opcional, con alineación a la derecha.
  \item[Índices:] Contenidos, figuras, tablas y códigos.
  \item[Introducción:] Importancia de la temática, vigencia y planteamiento del problema.
  \item[Marco teórico:] Fundamentos conceptuales y estado del arte.
  \item[Objetivos:] Objetivo general y específicos.
  \item[Metodología:] Tipo de investigación, técnicas y procedimientos.
  \item[Resultados:] Resultados obtenidos y análisis.
  \item[Conclusiones:] Resumen de objetivos conseguidos.
  \item[Bibliografía:] Referencias utilizadas (estilo APA recomendado).
  \item[Anexos:] Material complementario.
\end{description}

\section{Secciones y subsecciones}

En \LaTeX{} existen diferentes niveles de organización del contenido:

\begin{latexcode}[title={Niveles de secciones}]
\chapter{Capítulo}           % Nivel 0
\section{Sección}            % Nivel 1  
\subsection{Subsección}      % Nivel 2
\subsubsection{Subsubsección}% Nivel 3
\paragraph{Párrafo}          % Nivel 4
\end{latexcode}

\subsection{Ejemplo de subsección}

Este es el contenido de una subsección. Las subsecciones permiten organizar mejor el contenido dentro de cada sección.

\subsubsection{Ejemplo de subsubsección}

Las subsubsecciones son útiles para temas muy específicos dentro de una subsección.

\paragraph{Ejemplo de párrafo} Los párrafos con título son útiles para pequeñas divisiones que no necesitan aparecer en el índice.

\section{Citas bibliográficas}

Para citar la bibliografía según el sistema APA se utilizan los siguientes comandos. El archivo de bibliografía se encuentra en \texttt{referencias.bib}.

\begin{latexcode}[title={Comandos de citas biblatex/APA}]
% Cita textual: Autor (año)
\textcite{latex2024}

% Cita entre paréntesis: (Autor, año)
\parencite{latex2024}

% Cita con página específica
\parencite[Cap.~2]{latex2024}

% Múltiples citas
\parencite{latex2024,overleaf2024}
\end{latexcode}

Ejemplos de citas:
\begin{itemize}
  \item Cita textual: Según \textcite{latex2024}, \LaTeX{} es el estándar para documentos científicos.
  \item Cita entre paréntesis: \LaTeX{} es ampliamente utilizado \parencite{latex2024}.
  \item Múltiples fuentes: Existen varios recursos disponibles \parencite{latex2024,overleaf2024}.
\end{itemize}

\section{Notas al pie de página}

Las notas al pie se crean con el comando \verb|\footnote{texto}|.

La plantilla utiliza LuaLaTeX\footnote{LuaLaTeX es un motor de composición tipográfica que combina \LaTeX{} con el lenguaje de programación Lua, permitiendo mayor flexibilidad en el procesamiento de documentos.} como motor de compilación por sus capacidades avanzadas de manejo de fuentes\footnote{Permite usar cualquier fuente OpenType o TrueType instalada en el sistema sin necesidad de configuración adicional.}.

\section{Estilos de texto}

\LaTeX{} ofrece múltiples estilos de texto:

\begin{itemize}
  \item \textit{Texto en cursiva} -- \verb|\textit{texto}|
  \item \textbf{Texto en negrita} -- \verb|\textbf{texto}|
  \item \texttt{Texto monoespacio} -- \verb|\texttt{texto}|
  \item \textsc{Texto en versalitas} -- \verb|\textsc{texto}|
  \item \underline{Texto subrayado} -- \verb|\underline{texto}|
  \item \textbf{\textit{Negrita y cursiva}} -- combinación de comandos
  \item {\small Texto pequeño} -- \verb|{\small texto}|
  \item {\large Texto grande} -- \verb|{\large texto}|
\end{itemize}

\section{Acrónimos y glosario}

Los acrónimos se gestionan automáticamente con el paquete \texttt{glossaries}. La primera vez que aparece un acrónimo se muestra su forma completa, y en las siguientes apariciones solo la abreviatura.

\begin{latexcode}[title={Uso de acrónimos}]
% Primera aparición: muestra "Institute of Electrical 
% and Electronics Engineers (IEEE)"
El \gls{ieee} es una institución importante.

% Siguientes apariciones: muestra solo "IEEE"
El \gls{ieee} establece estándares.

% Para forzar forma larga: \glslong{ieee}
% Para forzar forma corta: \glsshort{ieee}
\end{latexcode}

Ejemplo: El \gls{ieee} es una institución importante en ingeniería. El \gls{ieee} establece estándares para la industria. Además del \gls{ieee}, existen otras organizaciones como la \gls{iso}.

Los acrónimos se definen en el archivo \texttt{contenido/anexos/acronimos.tex}.

\section{Tareas pendientes y notas}

Durante la redacción es útil marcar partes que requieren revisión usando el paquete \texttt{todonotes}:

\begin{latexcode}[title={Comandos de notas}]
% Nota al margen
\todo{Texto de la nota}

% Nota en línea
\todo[inline]{Nota dentro del texto}

% Figura pendiente
\missingfigure{Descripción de la figura}
\end{latexcode}

\todo{Esta es una nota al margen para recordar algo}

Las notas también pueden incluirse dentro del texto:

\todo[inline]{Esta es una nota en línea que indica trabajo pendiente para esta sección. Muy útil durante la fase de redacción.}

Para indicar figuras que faltan por añadir:

\missingfigure{Aquí irá un diagrama explicativo del flujo de trabajo}

\section{Comandos personalizados de la plantilla}

Esta plantilla proporciona comandos para acceder a la información configurada:

\begin{itemize}
  \item \verb|\EPStitulo| -- Título: ``\EPStitulo''
  \item \verb|\EPSautor| -- Autor: \EPSautor
  \item \verb|\EPStutor| -- Tutor: \EPStutor
  \item \verb|\EPSfecha| -- Fecha: \EPSfecha
  \item \verb|\EPStipoTrabajo| -- Tipo: \EPStipoTrabajo
  \item \verb|\EPStitulacion| -- Titulación: \EPStitulacion
\end{itemize}

Este documento es un \EPStipoTrabajo{} presentado en \EPSfecha.

\section{Hipervínculos y URLs}

Para incluir enlaces web se utiliza el comando \verb|\url{dirección}|:

\begin{itemize}
  \item Página de la EPS: \url{https://eps.ua.es}
  \item Overleaf (editor online): \url{https://www.overleaf.com}
  \item CTAN (repositorio de paquetes): \url{https://ctan.org}
\end{itemize}

Para enlaces con texto personalizado: \href{https://eps.ua.es}{Escuela Politécnica Superior}.

\section{Consejos para la redacción}

A continuación se presentan algunas recomendaciones para la redacción del \gls{tfg} o \gls{tfm}:

\subsection{Estructura del texto}

\begin{itemize}
  \item Utiliza párrafos cortos y concisos (máximo 8-10 líneas)
  \item Comienza cada capítulo con una breve introducción de su contenido
  \item Finaliza cada capítulo con un resumen o transición al siguiente
  \item Evita frases demasiado largas; divide las ideas complejas
\end{itemize}

\subsection{Estilo académico}

\begin{itemize}
  \item Escribe en tercera persona o forma impersonal (<<se ha desarrollado>> en lugar de <<he desarrollado>>)
  \item Mantén un tono formal y objetivo
  \item Evita expresiones coloquiales o jerga no técnica
  \item Define los términos técnicos la primera vez que aparecen
\end{itemize}

\subsection{Referencias y citas}

\begin{itemize}
  \item Cita siempre las fuentes de información que no sean tuyas
  \item Usa \verb|\parencite{clave}| para citas entre paréntesis: \parencite{knuth1984texbook}
  \item Usa \verb|\textcite{clave}| para citas narrativas: \textcite{lamport1994latex}
  \item Evita el exceso de citas; selecciona las más relevantes
  \item Verifica que todas las referencias del archivo \texttt{.bib} estén citadas
\end{itemize}

\subsection{Figuras y tablas}

\begin{itemize}
  \item Toda figura y tabla debe estar referenciada en el texto
  \item Las descripciones (captions) deben ser autoexplicativas
  \item Numera las figuras y tablas de forma consecutiva
  \item Usa formatos vectoriales (PDF, SVG) cuando sea posible
  \item Mantén consistencia en el estilo visual
\end{itemize}

\subsection{Revisión final}

Antes de entregar, verifica:

\begin{enumerate}
  \item Ortografía y gramática (usa correctores automáticos)
  \item Referencias cruzadas funcionan correctamente
  \item Bibliografía está completa y correctamente formateada
  \item Índices están actualizados
  \item Márgenes y formato cumplen la normativa
  \item No hay notas \texttt{\textbackslash todo} pendientes
\end{enumerate}
