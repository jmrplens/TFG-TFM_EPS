\chapter{Objetivos (Ejemplos de tablas)}
\label{ch:objetivos}

Este capítulo presenta los objetivos del trabajo y demuestra las diferentes formas de crear tablas en \LaTeX.

\section{Objetivo general}

El objetivo general de este trabajo es...

\section{Objetivos específicos}

\begin{enumerate}
  \item \textbf{OE1:} Analizar el estado actual de...
  \item \textbf{OE2:} Diseñar una arquitectura que permita...
  \item \textbf{OE3:} Implementar los componentes necesarios para...
  \item \textbf{OE4:} Evaluar el rendimiento del sistema mediante...
  \item \textbf{OE5:} Documentar el proceso de desarrollo y los resultados obtenidos.
\end{enumerate}

\section{Tablas en LaTeX}

Las tablas son elementos fundamentales en cualquier documento técnico. A continuación se muestran diferentes ejemplos.

\subsection{Tabla simple}

\begin{latexcode}[title={Código de tabla simple}]
\begin{table}[H]
  \centering
  \begin{tabular}{lcc}
           & Columna A & Columna B \\
    \hline
    Fila 1 & Dato 1A   & Dato 1B \\
    Fila 2 & Dato 2A   & Dato 2B \\
    Fila 3 & Dato 3A   & Dato 3B \\
    \hline
  \end{tabular}
  \caption{Ejemplo de tabla simple}
  \label{tab:simple}
\end{table}
\end{latexcode}

\begin{table}[H]
  \centering
  \begin{tabular}{lcc}
           & Columna A & Columna B \\
    \hline
    Fila 1 & Dato 1A   & Dato 1B \\
    Fila 2 & Dato 2A   & Dato 2B \\
    Fila 3 & Dato 3A   & Dato 3B \\
    \hline
  \end{tabular}
  \caption{Ejemplo de tabla simple}
  \label{tab:simple-ejemplo}
\end{table}

\subsection{Tabla con booktabs}

El paquete \texttt{booktabs} proporciona líneas más profesionales:

\begin{table}[H]
  \centering
  \caption{Tabla con estilo booktabs}
  \label{tab:booktabs}
  \begin{tabular}{@{}lcc@{}}
    \toprule
    \textbf{Concepto} & \textbf{Valor} & \textbf{Unidad} \\
    \midrule
    Velocidad máxima & 120 & km/h \\
    Consumo medio & 5.5 & L/100km \\
    Potencia & 150 & CV \\
    Par motor & 250 & Nm \\
    \bottomrule
  \end{tabular}
\end{table}

\subsection{Tabla con columnas de ancho fijo}

Puedes especificar el ancho de las columnas usando \texttt{p\{ancho\}}, \texttt{L\{ancho\}}, \texttt{C\{ancho\}} o \texttt{R\{ancho\}}:

\begin{table}[H]
  \centering
  \caption{Parámetros de posicionamiento de elementos flotantes}
  \label{tab:parametros}
  \begin{tabular}{|c|p{10cm}|}
    \hline
    \textbf{Parámetro} & \textbf{Significado} \\
    \hline
    \texttt{h} & Sitúa el elemento \emph{preferentemente} en la posición actual del texto \\
    \texttt{t} & Sitúa el elemento en la parte superior de la página \\
    \texttt{b} & Sitúa el elemento en la parte inferior de la página \\
    \texttt{p} & Sitúa el elemento en una página dedicada solo a flotantes \\
    \texttt{H} & Fuerza la posición exacta (requiere paquete \texttt{float}) \\
    \hline
  \end{tabular}
\end{table}

\subsection{Tabla con multicolumna y multifila}

\begin{table}[H]
  \centering
  \caption{Ejemplo de tabla con celdas combinadas}
  \label{tab:multicolrow}
  \begin{tabular}{|c|c|c|c|}
    \hline
    \multicolumn{4}{|c|}{\textbf{Distribución de recursos}} \\
    \hline
    \multirow{2}{*}{\textbf{Fase}} & \multicolumn{3}{c|}{\textbf{Recursos}} \\
    \cline{2-4}
     & Personal & Tiempo (h) & Coste (€) \\
    \hline
    Análisis & 2 & 40 & 2.000 \\
    Diseño & 3 & 80 & 4.800 \\
    Implementación & 4 & 160 & 9.600 \\
    Pruebas & 2 & 60 & 3.000 \\
    \hline
    \textbf{Total} & -- & \textbf{340} & \textbf{19.400} \\
    \hline
  \end{tabular}
\end{table}

\subsection{Tabla con colores alternados}

Usando el paquete \texttt{xcolor} con la opción \texttt{table}:

\begin{table}[H]
  \centering
  \caption{Especificaciones técnicas del sistema}
  \label{tab:specs}
  \rowcolors{2}{gray!15}{white}
  \begin{tabular}{@{}lcc@{}}
    \toprule
    \textbf{Componente} & \textbf{Especificación} & \textbf{Requisito} \\
    \midrule
    Procesador & Intel i7 / AMD Ryzen 7 & Mínimo \\
    Memoria RAM & 16 GB DDR4 & Recomendado \\
    Almacenamiento & 512 GB SSD & Mínimo \\
    GPU & NVIDIA RTX 3060 & Recomendado \\
    Sistema Operativo & Linux / Windows 11 & Compatible \\
    \bottomrule
  \end{tabular}
\end{table}

\subsection{Tabla larga (longtable)}

Para tablas que ocupan varias páginas, se usa el entorno \texttt{longtable}:

% Workaround for tagging error: caption outside longtable
\captionof{table}{Lista de requisitos del sistema}\label{tab:requisitos}
\begin{longtable}{@{}clp{7cm}@{}}
% \caption{Lista de requisitos del sistema} \\
  \toprule
  \textbf{ID} & \textbf{Tipo} & \textbf{Descripción} \\
  \midrule
  \endfirsthead
  
  \multicolumn{3}{c}{\tablename\ \thetable{} -- Continuación} \\
  \toprule
  \textbf{ID} & \textbf{Tipo} & \textbf{Descripción} \\
  \midrule
  \endhead
  
  \midrule
  \multicolumn{3}{r}{\textit{Continúa en la siguiente página}} \\
  \endfoot
  
  \bottomrule
  \endlastfoot
  
  RF01 & Funcional & El sistema debe permitir el registro de usuarios mediante correo electrónico \\
  RF02 & Funcional & Los usuarios deben poder iniciar sesión con sus credenciales \\
  RF03 & Funcional & El sistema debe generar informes en formato PDF \\
  RF04 & Funcional & Se debe implementar un sistema de notificaciones \\
  RF05 & Funcional & Los datos deben poder exportarse en formato CSV \\
  RNF01 & No funcional & El tiempo de respuesta no debe superar los 2 segundos \\
  RNF02 & No funcional & El sistema debe soportar 100 usuarios concurrentes \\
  RNF03 & No funcional & La disponibilidad debe ser del 99.5\% \\
  RNF04 & No funcional & La interfaz debe ser accesible según WCAG 2.1 \\
  RNF05 & No funcional & Los datos sensibles deben cifrarse con AES-256 \\
\end{longtable}

\section{Generadores de tablas}

Crear tablas manualmente puede ser tedioso. Se recomienda usar generadores online:

\begin{itemize}
  \item \url{https://www.tablesgenerator.com/} -- Generador visual muy completo
  \item \url{https://www.latex-tables.com/} -- Alternativa sencilla
  \item Excel2LaTeX -- Plugin para Microsoft Excel
\end{itemize}

\section{Forzar posición de tablas}

Para forzar que una tabla aparezca en un lugar específico, se puede usar:

\begin{latexcode}[title={Forzar posición}]
% Opción 1: Usar FloatBarrier
\begin{table}[H]
  ...
\end{table}
\FloatBarrier % Fuerza que la tabla aparezca antes

% Opción 2: Usar H (requiere paquete float)
\begin{table}[H]
  ...
\end{table}
\end{latexcode}

\section{Tablas con notas al pie}

Para añadir notas al pie dentro de una tabla, se usa el paquete \texttt{threeparttable}:

\begin{table}[H]
  \centering
  \caption{Comparativa de tecnologías web con notas}
  \label{tab:tecnologias-notas}
  \begin{threeparttable}
    \begin{tabular}{@{}lccc@{}}
      \toprule
      \textbf{Framework} & \textbf{Rendimiento}\tnote{a} & \textbf{Popularidad}\tnote{b} & \textbf{Curva} \\
      \midrule
      React     & 92 & 1st  & Media \\
      Vue.js    & 89 & 3rd  & Baja \\
      Angular   & 85 & 2nd  & Alta \\
      Svelte    & 96 & 5th  & Baja \\
      \bottomrule
    \end{tabular}
    \begin{tablenotes}
      \footnotesize
      \item[a] Puntuación en benchmarks de Lighthouse (0-100).
      \item[b] Posición según encuesta Stack Overflow 2024.
    \end{tablenotes}
  \end{threeparttable}
\end{table}

\section{Tabla con celdas de colores personalizados}

Se pueden colorear celdas individuales usando \texttt{\textbackslash cellcolor}:

\begin{table}[H]
  \centering
  \caption{Matriz de evaluación de riesgos}
  \label{tab:matriz-riesgos}
  \begin{tabular}{c|ccccc}
    \toprule
    \diagbox{Impacto}{Probabilidad} & Muy bajo & Bajo & Medio & Alto & Muy alto \\
    \midrule
    Muy alto & \cellcolor{yellow!50}M & \cellcolor{orange!50}A & \cellcolor{red!50}MA & \cellcolor{red!70}MA & \cellcolor{red!90}MA \\
    Alto     & \cellcolor{green!50}B & \cellcolor{yellow!50}M & \cellcolor{orange!50}A & \cellcolor{red!50}MA & \cellcolor{red!70}MA \\
    Medio    & \cellcolor{green!50}B & \cellcolor{green!30}B & \cellcolor{yellow!50}M & \cellcolor{orange!50}A & \cellcolor{red!50}MA \\
    Bajo     & \cellcolor{green!70}MB & \cellcolor{green!50}B & \cellcolor{green!30}B & \cellcolor{yellow!50}M & \cellcolor{orange!50}A \\
    Muy bajo & \cellcolor{green!70}MB & \cellcolor{green!70}MB & \cellcolor{green!50}B & \cellcolor{green!30}B & \cellcolor{yellow!50}M \\
    \bottomrule
  \end{tabular}
  
  \medskip
  \footnotesize
  MB = Muy Bajo, B = Bajo, M = Medio, A = Alto, MA = Muy Alto
\end{table}

\section{Tabla de ancho completo}

Para que una tabla ocupe todo el ancho disponible, se usa \texttt{tabularx}:

\begin{table}[H]
  \centering
  \caption{Cronograma del proyecto usando todo el ancho}
  \label{tab:cronograma}
  \begin{tabularx}{\textwidth}{@{}lXcccc@{}}
    \toprule
    \textbf{Fase} & \textbf{Descripción de actividades} & \textbf{M1} & \textbf{M2} & \textbf{M3} & \textbf{M4} \\
    \midrule
    Análisis & Recopilación de requisitos, entrevistas con stakeholders y estudio de viabilidad & $\bullet$ & & & \\
    Diseño & Arquitectura del sistema, diseño de base de datos y prototipado de interfaz & $\bullet$ & $\bullet$ & & \\
    Desarrollo & Implementación del backend, frontend y pruebas unitarias & & $\bullet$ & $\bullet$ & \\
    Testing & Pruebas de integración, UAT y corrección de errores & & & $\bullet$ & $\bullet$ \\
    Despliegue & Configuración de servidores, migración y formación de usuarios & & & & $\bullet$ \\
    \bottomrule
  \end{tabularx}
\end{table}
