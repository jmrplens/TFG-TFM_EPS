\chapter{Objetivos (Ejemplos de tablas)}
\label{ch:objetivos}

Este capítulo presenta los objetivos del trabajo y demuestra las diferentes formas de crear tablas en \LaTeX{}. Las tablas son \glspl{flotante} fundamentales para presentar datos de forma estructurada, típicamente obtenidos de \glspl{query} a bases de datos.

\section{Objetivo general}
\label{sec:objetivo-general}

El objetivo general de este trabajo es desarrollar un sistema con alta \gls{escalabilidad}...

\section{Objetivos específicos}
\label{sec:objetivos-especificos}

\begin{enumerate}
  \item \textbf{OE1:} Analizar el estado actual usando \glspl{algoritmo} de búsqueda...
  \item \textbf{OE2:} Diseñar una arquitectura basada en \glspl{microservicio} que permita...
  \item \textbf{OE3:} Implementar el \gls{backend} y \gls{frontend} necesarios para...
  \item \textbf{OE4:} Evaluar el rendimiento del sistema mediante pruebas de \gls{latencia}...
  \item \textbf{OE5:} Documentar el proceso de desarrollo y los resultados obtenidos.
\end{enumerate}

\section{Tablas en LaTeX}
\label{sec:tablas-latex}

Las tablas son elementos flotantes en \LaTeX{}, al igual que las figuras. A continuación se muestran diferentes ejemplos y técnicas para crearlas.

\subsection{Estructura básica de una tabla}
\label{subsec:estructura-tabla}

La estructura básica de una tabla en \LaTeX{} es:

\begin{verbatim}
\begin{table}[posición]
  \centering
  \caption{Título de la tabla}
  \label{tab:etiqueta}
  \begin{tabular}{especificación-columnas}
    ... contenido ...
  \end{tabular}
\end{table}
\end{verbatim}

Donde la especificación de columnas puede ser:
\begin{itemize}
  \item \texttt{l} --- Alineación izquierda
  \item \texttt{c} --- Alineación centrada
  \item \texttt{r} --- Alineación derecha
  \item \texttt{p\{ancho\}} --- Columna con ancho fijo y texto justificado
  \item \texttt{|} --- Línea vertical entre columnas
\end{itemize}

\subsection{Tabla simple}
\label{subsec:tabla-simple}

\begin{verbatim}
\begin{table}[H]
  \centering
  \begin{tabular}{lcc}
           & Columna A & Columna B \\
    \hline
    Fila 1 & Dato 1A   & Dato 1B \\
    Fila 2 & Dato 2A   & Dato 2B \\
    \hline
  \end{tabular}
  \caption{Ejemplo de tabla simple}
  \label{tab:ejemplo}
\end{table}
\end{verbatim}

\begin{table}[H]
  \centering
  \begin{tabular}{lcc}
           & Columna A & Columna B \\
    \hline
    Fila 1 & Dato 1A   & Dato 1B \\
    Fila 2 & Dato 2A   & Dato 2B \\
    Fila 3 & Dato 3A   & Dato 3B \\
    \hline
  \end{tabular}
  \caption{Ejemplo de tabla simple}
  \label{tab:simple-ejemplo}
\end{table}

\subsection{Tabla con booktabs (recomendado)}
\label{subsec:tabla-booktabs}

El paquete \texttt{booktabs} proporciona líneas horizontales más elegantes y profesionales. Esta plantilla ya lo incluye.

\begin{verbatim}
\begin{table}[H]
  \centering
  \caption{Tabla con estilo booktabs}
  \begin{tabular}{@{}lcc@{}}
    \toprule
    \textbf{Concepto} & \textbf{Valor} & \textbf{Unidad} \\
    \midrule
    Velocidad máxima & 120 & km/h \\
    Consumo medio & 5.5 & L/100km \\
    \bottomrule
  \end{tabular}
\end{table}
\end{verbatim}

\begin{infobox}{Recomendación de estilo}
Usa siempre \texttt{booktabs} para tablas profesionales. Evita líneas verticales y usa solo \verb|\toprule|, \verb|\midrule| y \verb|\bottomrule|.
\end{infobox}

\begin{table}[H]
  \centering
  \caption{Tabla con estilo booktabs}
  \label{tab:booktabs}
  \begin{tabular}{@{}lcc@{}}
    \toprule
    \textbf{Concepto} & \textbf{Valor} & \textbf{Unidad} \\
    \midrule
    Velocidad máxima & 120 & km/h \\
    Consumo medio & 5.5 & L/100km \\
    Potencia & 150 & CV \\
    Par motor & 250 & Nm \\
    \bottomrule
  \end{tabular}
\end{table}

\subsection{Tabla con columnas de ancho fijo}
\label{subsec:tabla-ancho-fijo}

Puedes especificar el ancho de las columnas usando \texttt{p\{ancho\}}:

\begin{verbatim}
% Columnas de ancho fijo con alineación
p{5cm}  % Justificado, ancho 5cm
m{5cm}  % Centrado verticalmente
b{5cm}  % Alineado abajo
\end{verbatim}

\begin{table}[H]
  \centering
  \caption{Parámetros de posicionamiento de elementos flotantes}
  \label{tab:parametros}
  \begin{tabular}{|c|p{10cm}|}
    \hline
    \textbf{Parámetro} & \textbf{Significado} \\
    \hline
    \texttt{h} & Sitúa el elemento \emph{preferentemente} en la posición actual del texto \\
    \texttt{t} & Sitúa el elemento en la parte superior de la página \\
    \texttt{b} & Sitúa el elemento en la parte inferior de la página \\
    \texttt{p} & Sitúa el elemento en una página dedicada solo a flotantes \\
    \texttt{H} & Fuerza la posición exacta (requiere paquete \texttt{float}) \\
    \hline
  \end{tabular}
\end{table}

\subsection{Tabla con multicolumna y multifila}
\label{subsec:tabla-multicolrow}

Para combinar celdas se recomienda usar el paquete moderno \texttt{tabularray} que evita conflictos con otros paquetes:

\begin{verbatim}
% Sintaxis tabularray para combinar celdas
\begin{tblr}{
  colspec = {llccc},
  cell{2}{1} = {r=2}{},  % Combina 2 filas
  cell{4}{1} = {c=3}{},  % Combina 3 columnas
}
  Cabecera & Col2 & Col3 & Col4 & Col5 \\
  Fase A   & Tarea 1 & 1 & 10 & 100 \\
           & Tarea 2 & 2 & 20 & 200 \\
  Total    &         &   & 30 & 300 \\
\end{tblr}
\end{verbatim}

\begin{table}[H]
  \centering
  \caption{Ejemplo de tabla con celdas combinadas (tabularray)}
  \label{tab:multicolrow}
  \begin{tblr}{
    colspec = {llccc},
    row{1} = {font=\bfseries},
    row{6} = {font=\bfseries},
    hline{1} = {1pt},
    hline{2,4,6,7} = {0.5pt},
    cell{2}{1} = {r=2}{},
    cell{4}{1} = {r=2}{},
    cell{6}{1} = {c=3}{},
  }
    Fase     & Tarea        & Personal & Tiempo (h) & Coste (€) \\
    Análisis & Requisitos   & 2        & 20         & 1.000     \\
             & Arquitectura & 2        & 40         & 2.400     \\
    Diseño   & Prototipado  & 2        & 40         & 2.400     \\
             & UI/UX        & 1        & 30         & 1.800     \\
    Total    &              &          & 130        & 7.600     \\
  \end{tblr}
\end{table}

\subsection{Tabla con colores alternados}
\label{subsec:tabla-colores-alternados}

Usando el paquete \texttt{xcolor} con la opción \texttt{table}:

\begin{verbatim}
% Activar colores alternados
\rowcolors{2}{gray!15}{white}
\begin{tabular}{...}
  ... contenido ...
\end{tabular}
\end{verbatim}

\begin{table}[H]
  \centering
  \caption{Especificaciones técnicas del sistema}
  \label{tab:specs}
  \rowcolors{2}{gray!15}{white}
  \begin{tabular}{@{}lcc@{}}
    \toprule
    \textbf{Componente} & \textbf{Especificación} & \textbf{Requisito} \\
    \midrule
    Procesador & Intel i7 / AMD Ryzen 7 & Mínimo \\
    Memoria RAM & 16 GB DDR4 & Recomendado \\
    Almacenamiento & 512 GB SSD & Mínimo \\
    GPU & NVIDIA RTX 3060 & Recomendado \\
    Sistema Operativo & Linux / Windows 11 & Compatible \\
    \bottomrule
  \end{tabular}
\end{table}



\section{Generadores de tablas}
\label{sec:generadores-tablas}

Crear tablas manualmente puede ser tedioso. Se recomienda usar generadores online:

\begin{itemize}
  \item \url{https://www.tablesgenerator.com/} --- Generador visual muy completo
  \item \url{https://www.latex-tables.com/} --- Alternativa sencilla
  \item Excel2LaTeX --- Plugin para Microsoft Excel
\end{itemize}

\begin{tipbox}{Flujo de trabajo recomendado}
Crea la estructura básica con un generador online, luego ajusta manualmente el estilo usando \texttt{booktabs}.
\end{tipbox}

\section{Forzar posición de tablas}
\label{sec:forzar-posicion}

Para forzar que una tabla aparezca en un lugar específico:

\begin{verbatim}
% Opción 1: Usar el especificador H
\begin{table}[H]  % Requiere paquete float
  ...
\end{table}

% Opción 2: Usar FloatBarrier
\begin{table}[htbp]
  ...
\end{table}
\FloatBarrier  % Fuerza que aparezca antes de continuar
\end{verbatim}

\section{Tablas con notas al pie}
\label{sec:tablas-notas}

Para añadir notas al pie dentro de una tabla, se usa el paquete \texttt{threeparttable}:

\begin{verbatim}
\begin{threeparttable}
  \begin{tabular}{...}
    ... Texto\tnote{a} ...
  \end{tabular}
  \begin{tablenotes}
    \item[a] Explicación de la nota.
  \end{tablenotes}
\end{threeparttable}
\end{verbatim}

\begin{table}[H]
  \centering
  \caption{Comparativa de tecnologías web con notas}
  \label{tab:tecnologias-notas}
  \begin{threeparttable}
    \begin{tabular}{@{}lccc@{}}
      \toprule
      \textbf{Framework} & \textbf{Rendimiento}\tnote{a} & \textbf{Popularidad}\tnote{b} & \textbf{Curva} \\
      \midrule
      React     & 92 & 1st  & Media \\
      Vue.js    & 89 & 3rd  & Baja \\
      Angular   & 85 & 2nd  & Alta \\
      Svelte    & 96 & 5th  & Baja \\
      \bottomrule
    \end{tabular}
    \begin{tablenotes}
      \footnotesize
      \item[a] Puntuación en benchmarks de Lighthouse (0-100).
      \item[b] Posición según encuesta Stack Overflow 2024.
    \end{tablenotes}
  \end{threeparttable}
\end{table}

\section{Tabla con celdas de colores personalizados}
\label{sec:tablas-colores}

Se pueden colorear celdas individuales usando \verb|\cellcolor{color}|:

\begin{verbatim}
% Celda coloreada
\cellcolor{red!50} Texto con fondo rojo

% Fila completa coloreada
\rowcolor{blue!20}

% Columna coloreada (en especificación)
>{\columncolor{gray!20}}c
\end{verbatim}

\begin{table}[H]
  \centering
  \caption{Matriz de evaluación de riesgos}
  \label{tab:matriz-riesgos}
  \begin{tabular}{c|ccccc}
    \toprule
    \diagbox{Impacto}{Probabilidad} & Muy bajo & Bajo & Medio & Alto & Muy alto \\
    \midrule
    Muy alto & \cellcolor{yellow!50}M & \cellcolor{orange!50}A & \cellcolor{red!50}MA & \cellcolor{red!70}MA & \cellcolor{red!90}MA \\
    Alto     & \cellcolor{green!50}B & \cellcolor{yellow!50}M & \cellcolor{orange!50}A & \cellcolor{red!50}MA & \cellcolor{red!70}MA \\
    Medio    & \cellcolor{green!50}B & \cellcolor{green!30}B & \cellcolor{yellow!50}M & \cellcolor{orange!50}A & \cellcolor{red!50}MA \\
    Bajo     & \cellcolor{green!70}MB & \cellcolor{green!50}B & \cellcolor{green!30}B & \cellcolor{yellow!50}M & \cellcolor{orange!50}A \\
    Muy bajo & \cellcolor{green!70}MB & \cellcolor{green!70}MB & \cellcolor{green!50}B & \cellcolor{green!30}B & \cellcolor{yellow!50}M \\
    \bottomrule
  \end{tabular}

  \medskip
  \footnotesize
  MB = Muy Bajo, B = Bajo, M = Medio, A = Alto, MA = Muy Alto
\end{table}

\section{Tabla de ancho completo}
\label{sec:tablas-ancho-completo}

Para que una tabla ocupe todo el ancho disponible, se usa \texttt{tabularx}:

\begin{verbatim}
\begin{tabularx}{\textwidth}{@{}lXcc@{}}
  % X = columna que se expande
  % l, c, r = columnas normales
\end{tabularx}
\end{verbatim}

\begin{table}[H]
  \centering
  \caption{Cronograma del proyecto usando todo el ancho}
  \label{tab:cronograma}
  \begin{tabularx}{\textwidth}{@{}lXcccc@{}}
    \toprule
    \textbf{Fase} & \textbf{Descripción de actividades} & \textbf{M1} & \textbf{M2} & \textbf{M3} & \textbf{M4} \\
    \midrule
    Análisis & Recopilación de requisitos, entrevistas con stakeholders y estudio de viabilidad & $\bullet$ & & & \\
    Diseño & Arquitectura del sistema, diseño de base de datos y prototipado de interfaz & $\bullet$ & $\bullet$ & & \\
    Desarrollo & Implementación del backend, frontend y pruebas unitarias & & $\bullet$ & $\bullet$ & \\
    Testing & Pruebas de integración, UAT y corrección de errores & & & $\bullet$ & $\bullet$ \\
    Despliegue & Configuración de servidores, migración y formación de usuarios & & & & $\bullet$ \\
    \bottomrule
  \end{tabularx}
\end{table}

\section{Resumen de comandos para tablas}
\label{sec:resumen-comandos-tablas}

A modo de referencia rápida:

\begin{verbatim}
% Estructura básica
\begin{table}[H]
  \centering
  \caption{Título}
  \label{tab:etiqueta}
  \begin{tabular}{@{}lcc@{}}
    \toprule
    Cabecera 1 & Cabecera 2 & Cabecera 3 \\
    \midrule
    Dato 1 & Dato 2 & Dato 3 \\
    \bottomrule
  \end{tabular}
\end{table}

% Comandos de booktabs
\toprule      % Línea superior gruesa
\midrule      % Línea intermedia
\bottomrule   % Línea inferior gruesa
\cmidrule{a-b}% Línea parcial de columna a hasta b

% Combinar celdas
\multicolumn{n}{alineación}{texto}
\multirow{n}{ancho}{texto}

% Colores
\rowcolors{inicio}{color1}{color2}  % Filas alternadas
\cellcolor{color}                    % Celda individual
\rowcolor{color}                     % Fila completa
\end{verbatim}
