\chapter{Conclusiones (Ejemplos de matemáticas)}
\label{ch:conclusiones}

Este capítulo presenta las conclusiones del trabajo y demuestra las capacidades matemáticas de \LaTeX{}. Se incluyen ejemplos de \glspl{algoritmo} expresados en notación matemática.

\section{Ecuaciones matemáticas}
\label{sec:ecuaciones-matematicas}

\LaTeX{} es el estándar para la composición de fórmulas matemáticas. A continuación se muestran diferentes formas de escribir matemáticas usando diversos \glspl{entorno}.

\subsection{Modos matemáticos}
\label{subsec:modos-matematicos}

Existen dos modos de escribir matemáticas:

\begin{verbatim}
% Modo inline (en línea con el texto)
La ecuación $E = mc^2$ es famosa.

% Modo display (centrado, en línea aparte)
\[ E = mc^2 \]

% También con doble dólar (no recomendado)
$$ E = mc^2 $$
\end{verbatim}

\begin{infobox}{Recomendación}
Usa \verb|$...$| para fórmulas en línea y \verb|\[...\]| o el entorno \texttt{equation} para fórmulas en modo display. Evita \verb|$$...$$| ya que puede causar problemas de espaciado.
\end{infobox}

\subsection{Ecuaciones numeradas}
\label{subsec:ecuaciones-numeradas}

Para mostrar una ecuación numerada se usa el entorno \texttt{equation}:

\begin{verbatim}
\begin{equation}
  E = mc^2
  \label{eq:einstein}
\end{equation}

Referencia: como se ve en la Ecuación \ref{eq:einstein}...
\end{verbatim}

\begin{equation}
  E = mc^2
  \label{eq:einstein-demo}
\end{equation}

La famosa ecuación de Einstein (\ref{eq:einstein-demo}) relaciona masa y energía.

\subsection{Fracciones y raíces}
\label{subsec:fracciones-raices}

\begin{verbatim}
% Fracciones
\frac{numerador}{denominador}
\dfrac{a}{b}  % Fuerza tamaño display
\tfrac{a}{b}  % Fuerza tamaño texto

% Raíces
\sqrt{x}      % Raíz cuadrada
\sqrt[3]{x}   % Raíz cúbica
\sqrt[n]{x}   % Raíz n-ésima
\end{verbatim}

\begin{equation}
  x = \frac{-b \pm \sqrt{b^2 - 4ac}}{2a}
  \label{eq:cuadratica}
\end{equation}

\begin{equation}
  \sqrt[3]{\frac{a^3 + b^3}{2}} \neq \frac{\sqrt[3]{a^3} + \sqrt[3]{b^3}}{2}
  \label{eq:raices}
\end{equation}

\subsection{Sumas, integrales y límites}
\label{subsec:sumas-integrales}

\begin{verbatim}
% Sumatorios
\sum_{i=1}^{n} a_i         % Suma de i=1 a n
\prod_{i=1}^{n} a_i        % Productorio

% Integrales
\int_{a}^{b} f(x) \, dx    % Integral definida
\iint_S f \, dA            % Integral doble
\oint_C f \, dz            % Integral de contorno

% Límites
\lim_{x \to \infty} f(x)   % Límite
\end{verbatim}

\begin{equation}
  \int_{-\infty}^{\infty} e^{-x^2} dx = \sqrt{\pi}
  \label{eq:gauss}
\end{equation}

\begin{equation}
  \sum_{n=1}^{\infty} \frac{1}{n^2} = \frac{\pi^2}{6}
  \label{eq:basel}
\end{equation}

\begin{equation}
  \lim_{n \to \infty} \left(1 + \frac{1}{n}\right)^n = e
  \label{eq:euler}
\end{equation}

\subsection{Operadores diferenciales}
\label{subsec:operadores-diferenciales}

\begin{verbatim}
% Derivadas
\frac{dy}{dx}              % Derivada
\frac{\partial f}{\partial x}  % Derivada parcial
\frac{d^2 y}{dx^2}         % Segunda derivada

% Operadores vectoriales
\nabla f                   % Gradiente
\nabla \cdot \mathbf{F}    % Divergencia
\nabla \times \mathbf{F}   % Rotacional
\end{verbatim}

\begin{equation}
  \nabla \times \mathbf{H} = \left[ \frac{1}{r} \frac{\partial}{\partial r}(r H_\theta) - \frac{1}{r} \frac{\partial H_r}{\partial \theta} \right] \hat{\mathbf{z}}
  \label{eq:maxwell}
\end{equation}

\subsection{Sistemas de ecuaciones}
\label{subsec:sistemas-ecuaciones}

\begin{verbatim}
\begin{equation}
  \begin{cases}
    x + y + z = 6 \\
    2x - y + z = 3 \\
    x + 2y - z = 2
  \end{cases}
\end{equation}
\end{verbatim}

\begin{equation}
  \begin{cases}
    x + y + z = 6 \\
    2x - y + z = 3 \\
    x + 2y - z = 2
  \end{cases}
  \label{eq:sistema}
\end{equation}

\subsection{Ecuaciones alineadas}
\label{subsec:ecuaciones-alineadas}

El entorno \texttt{align} permite alinear múltiples ecuaciones por el símbolo \texttt{\&}:

\begin{verbatim}
\begin{align}
  (a+b)^2 &= a^2 + 2ab + b^2 \\
  (a-b)^2 &= a^2 - 2ab + b^2 \\
  (a+b)(a-b) &= a^2 - b^2
\end{align}

% Sin números: usar align*
\begin{align*}
  ...
\end{align*}
\end{verbatim}

\begin{align}
  (a+b)^2 &= a^2 + 2ab + b^2 \\
  (a-b)^2 &= a^2 - 2ab + b^2 \\
  (a+b)(a-b) &= a^2 - b^2
\end{align}

\subsection{Subecuaciones}
\label{subsec:subecuaciones}

Para numerar ecuaciones como (1a), (1b), usa \texttt{subequations}:

\begin{verbatim}
\begin{subequations}
  \begin{align}
    \mathbf{E} &= E_z \hat{\mathbf{z}} \label{eq:campo-E} \\
    \mathbf{H} &= H_r \hat{\mathbf{r}} \label{eq:campo-H}
  \end{align}
\end{subequations}
\end{verbatim}

\begin{subequations}
  \begin{align}
    \mathbf{E} &= E_z(r,\theta) \hat{\mathbf{z}} \label{eq:E} \\
    \mathbf{H} &= H_r(r,\theta) \hat{\mathbf{r}} + H_\theta(r,\theta) \hat{\theta} \label{eq:H}
  \end{align}
\end{subequations}

Las ecuaciones (\ref{eq:E}) y (\ref{eq:H}) describen los campos electromagnéticos.

\subsection{Matrices}
\label{subsec:matrices}

\begin{verbatim}
% Tipos de delimitadores
\begin{pmatrix} ... \end{pmatrix}  % Paréntesis ( )
\begin{bmatrix} ... \end{bmatrix}  % Corchetes [ ]
\begin{vmatrix} ... \end{vmatrix}  % Barras | | (determinante)
\begin{Bmatrix} ... \end{Bmatrix}  % Llaves { }

% Ejemplo
\begin{pmatrix}
  a & b \\
  c & d
\end{pmatrix}
\end{verbatim}

\begin{equation}
  \mathbf{A} = \begin{pmatrix}
    a_{11} & a_{12} & a_{13} \\
    a_{21} & a_{22} & a_{23} \\
    a_{31} & a_{32} & a_{33}
  \end{pmatrix}
  \label{eq:matriz}
\end{equation}

\begin{equation}
  \det(\mathbf{A}) = \begin{vmatrix}
    a & b \\
    c & d
  \end{vmatrix} = ad - bc
  \label{eq:determinante}
\end{equation}

\subsection{Símbolos griegos y especiales}
\label{subsec:simbolos-griegos}

\begin{verbatim}
% Letras griegas minúsculas
\alpha \beta \gamma \delta \epsilon \zeta \eta \theta
\iota \kappa \lambda \mu \nu \xi \pi \rho
\sigma \tau \upsilon \phi \chi \psi \omega

% Letras griegas mayúsculas
\Gamma \Delta \Theta \Lambda \Xi \Pi \Sigma \Phi \Psi \Omega

% Símbolos de relación
\leq \geq \neq \approx \equiv \sim \propto \in \subset

% Operadores
\times \div \pm \mp \cdot \cap \cup \wedge \vee
\end{verbatim}

Ejemplo: $\alpha + \beta = \gamma$, $\forall x \in \mathbb{R}: x^2 \geq 0$

\subsection{Ecuación con condiciones}
\label{subsec:ecuacion-condiciones}

La función \texttt{condiciones} de la plantilla permite documentar las variables:

\begin{equation}
  \underset{z=z_0}{\mathrm{Res}}(f(z)) = \frac{1}{(m-1)!} \lim_{z \to z_0} \left[ \frac{d^{m-1}}{dz^{m-1}} \left[ (z-z_0)^m f(z) \right] \right]
  \label{eq:residuo}
\end{equation}

\begin{condiciones}[donde:]
  m & \rightarrow & es la multiplicidad del polo $z_0$ \\
  z_0 & \rightarrow & es el punto donde se calcula el residuo \\
  f(z) & \rightarrow & es la función analítica
\end{condiciones}

\subsection{Teoremas y demostraciones}
\label{subsec:teoremas-demostraciones}

La plantilla incluye entornos para teoremas, demostraciones y definiciones:

\begin{verbatim}
\begin{theorem}[Nombre del teorema]
  Enunciado del teorema...
\end{theorem}

\begin{proof}
  Demostración... \qedhere
\end{proof}

\begin{definition}[Nombre]
  Definición formal...
\end{definition}
\end{verbatim}

\begin{theorem}[Teorema de Pitágoras]
  En un triángulo rectángulo, el cuadrado de la hipotenusa es igual a la suma de los cuadrados de los catetos:
  \begin{equation}
    c^2 = a^2 + b^2
  \end{equation}
\end{theorem}

\begin{proof}
  Sea un triángulo rectángulo con catetos $a$ y $b$, e hipotenusa $c$. Considerando el área del cuadrado de lado $(a+b)$...

  Por lo tanto, $c^2 = a^2 + b^2$. \qedhere
\end{proof}

\begin{definition}[Límite]
  Sea $f: D \to \mathbb{R}$ una función y $a$ un punto de acumulación de $D$. Decimos que $L$ es el límite de $f$ en $a$ si:
  \begin{equation}
    \forall \varepsilon > 0, \exists \delta > 0 : 0 < |x-a| < \delta \Rightarrow |f(x) - L| < \varepsilon
  \end{equation}
\end{definition}

\section{Conclusiones generales}
\label{sec:conclusiones-generales}

El desarrollo de este \EPStipoTrabajo{} ha permitido alcanzar los objetivos planteados inicialmente. Las principales conclusiones son:

\begin{enumerate}
  \item Se ha desarrollado con éxito un sistema que cumple con los requisitos funcionales y no funcionales establecidos.

  \item La arquitectura diseñada ha demostrado ser escalable y mantenible, facilitando futuras extensiones del sistema.

  \item Los resultados de las pruebas confirman que el sistema es robusto y ofrece un rendimiento adecuado para el caso de uso planteado.

  \item El proceso de desarrollo ha seguido las mejores prácticas de ingeniería de software, incluyendo control de versiones, integración continua y documentación exhaustiva.
\end{enumerate}

\section{Aportaciones del trabajo}
\label{sec:aportaciones-trabajo}

Las principales aportaciones de este trabajo son:

\begin{description}
  \item[Aportación técnica:] Desarrollo de una solución innovadora que mejora en un 40\% el rendimiento respecto a sistemas anteriores.

  \item[Aportación metodológica:] Aplicación de una metodología híbrida que combina prácticas ágiles con documentación formal.

  \item[Aportación práctica:] Sistema funcional desplegado y en uso por usuarios reales.
\end{description}

\section{Dificultades encontradas}
\label{sec:dificultades-encontradas}

Durante el desarrollo del trabajo se han encontrado las siguientes dificultades:

\begin{enumerate}
  \item \textbf{Integración de sistemas:} La integración con sistemas externos requirió un esfuerzo adicional debido a diferencias en los formatos de datos.

  \item \textbf{Rendimiento:} Fue necesario optimizar varias consultas a base de datos para cumplir con los requisitos de tiempo de respuesta.

  \item \textbf{Documentación externa:} Parte de la documentación de las bibliotecas utilizadas estaba desactualizada.
\end{enumerate}

\section{Trabajo futuro}
\label{sec:trabajo-futuro}

Como líneas de trabajo futuro se proponen:

\begin{enumerate}
  \item \textbf{Ampliación de funcionalidades:} Implementar las características identificadas como deseables pero fuera del alcance inicial.

  \item \textbf{Mejora del rendimiento:} Implementar un sistema de caché distribuida para mejorar los tiempos de respuesta en escenarios de alta carga.

  \item \textbf{Versión móvil:} Desarrollar una aplicación móvil nativa para iOS y Android.

  \item \textbf{Internacionalización:} Añadir soporte para múltiples idiomas en la interfaz de usuario.

  \item \textbf{Machine Learning:} Incorporar algoritmos de aprendizaje automático para predicción y recomendaciones.
\end{enumerate}

\section{Valoración personal}
\label{sec:valoracion-personal}

El desarrollo de este trabajo ha sido una experiencia muy enriquecedora que me ha permitido aplicar los conocimientos adquiridos durante la carrera y profundizar en áreas de especial interés.

Entre los aprendizajes más valiosos destacan:

\begin{itemize}
  \item La importancia de una buena planificación inicial
  \item El valor de las pruebas automatizadas para garantizar la calidad
  \item La necesidad de documentar adecuadamente el código
  \item Las habilidades de comunicación y gestión de proyectos
\end{itemize}

Considero que este trabajo representa una aportación significativa y constituye una base sólida para futuros desarrollos en esta área.
