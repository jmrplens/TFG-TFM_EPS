\chapter{Conclusiones}
\label{ch:conclusiones}

Este capítulo presenta las conclusiones del trabajo realizado, así como las posibles líneas de trabajo futuro.

\section{Conclusiones generales}

El desarrollo de este \EPStipoTrabajo{} ha permitido alcanzar los objetivos planteados inicialmente. Las principales conclusiones son:

\begin{enumerate}
  \item Se ha desarrollado con éxito un sistema que cumple con los requisitos funcionales y no funcionales establecidos.
  
  \item La arquitectura diseñada ha demostrado ser escalable y mantenible, facilitando futuras extensiones.
  
  \item Los resultados de las pruebas confirman que el sistema es robusto y ofrece un rendimiento adecuado.
  
  \item El proceso de desarrollo ha seguido las mejores prácticas de ingeniería de software.
\end{enumerate}

\section{Aportaciones del trabajo}

Las principales aportaciones de este trabajo son:

\begin{itemize}
  \item \textbf{Aportación técnica:} Desarrollo de una solución innovadora para...
  
  \item \textbf{Aportación metodológica:} Aplicación de una metodología adaptada a...
  
  \item \textbf{Aportación práctica:} Sistema funcional que puede ser utilizado por...
\end{itemize}

\section{Dificultades encontradas}

Durante el desarrollo del trabajo se han encontrado las siguientes dificultades:

\begin{enumerate}
  \item \textbf{Integración de sistemas:} La integración con sistemas externos requirió un esfuerzo adicional debido a...
  
  \item \textbf{Rendimiento:} Fue necesario optimizar varias consultas para cumplir con los requisitos de tiempo de respuesta.
  
  \item \textbf{Documentación externa:} Parte de la documentación de las herramientas utilizadas estaba desactualizada.
\end{enumerate}

\section{Trabajo futuro}

Como líneas de trabajo futuro se proponen:

\begin{enumerate}
  \item \textbf{Ampliación de funcionalidades:} Implementar las funcionalidades identificadas como deseables pero fuera del alcance inicial.
  
  \item \textbf{Mejora del rendimiento:} Implementar un sistema de caché distribuida para mejorar los tiempos de respuesta.
  
  \item \textbf{Versión móvil:} Desarrollar una aplicación móvil nativa para iOS y Android.
  
  \item \textbf{Internacionalización:} Añadir soporte para múltiples idiomas.
  
  \item \textbf{Machine Learning:} Incorporar algoritmos de aprendizaje automático para...
\end{enumerate}

\section{Valoración personal}

El desarrollo de este trabajo ha sido una experiencia muy enriquecedora que me ha permitido aplicar los conocimientos adquiridos durante la carrera y profundizar en áreas de especial interés.

Entre los aprendizajes más valiosos destacan:

\begin{itemize}
  \item La importancia de una buena planificación inicial
  \item El valor de las pruebas automatizadas
  \item La necesidad de documentar adecuadamente el código
  \item Las habilidades de comunicación y gestión de proyectos
\end{itemize}

Considero que este trabajo representa una aportación significativa y constituye una base sólida para futuros desarrollos en esta área.
