%%%%%%%%%%%%%%%%%%%%%%%%%%%%%%%%%%%%%%%%%%%%%%%%%%%%%%%%%%%%%%%%%%%%%%%%
% ACRÓNIMOS Y ABREVIATURAS
%
% Formato: \newacronym{etiqueta}{siglas}{descripción}
% Uso en el texto: \gls{etiqueta} o \acrlong{etiqueta} o \acrshort{etiqueta}
%%%%%%%%%%%%%%%%%%%%%%%%%%%%%%%%%%%%%%%%%%%%%%%%%%%%%%%%%%%%%%%%%%%%%%%%

% === Acrónimos generales ===
\newacronym{tfg}{TFG}{Trabajo Fin de Grado}
\newacronym{tfm}{TFM}{Trabajo Fin de Máster}
\newacronym{eps}{EPS}{Escuela Politécnica Superior}
\newacronym{ua}{UA}{Universidad de Alicante}

% === Organizaciones y estándares ===
\newacronym{ieee}{IEEE}{Institute of Electrical and Electronics Engineers}
\newacronym{iso}{ISO}{International Organization for Standardization}
\newacronym{apa}{APA}{American Psychological Association}
\newacronym{w3c}{W3C}{World Wide Web Consortium}
\newacronym{ietf}{IETF}{Internet Engineering Task Force}

% === Acrónimos técnicos ===
\newacronym{api}{API}{Application Programming Interface}
\newacronym{rest}{REST}{Representational State Transfer}
\newacronym{sql}{SQL}{Structured Query Language}
\newacronym{html}{HTML}{HyperText Markup Language}
\newacronym{css}{CSS}{Cascading Style Sheets}
\newacronym{json}{JSON}{JavaScript Object Notation}
\newacronym{xml}{XML}{eXtensible Markup Language}
\newacronym{http}{HTTP}{HyperText Transfer Protocol}
\newacronym{https}{HTTPS}{HyperText Transfer Protocol Secure}
\newacronym{url}{URL}{Uniform Resource Locator}
\newacronym{uri}{URI}{Uniform Resource Identifier}

% === Acrónimos de desarrollo ===
\newacronym{ci}{CI}{Continuous Integration}
\newacronym{cd}{CD}{Continuous Delivery}
\newacronym{tdd}{TDD}{Test-Driven Development}
\newacronym{mvc}{MVC}{Model-View-Controller}
\newacronym{mvvm}{MVVM}{Model-View-ViewModel}
\newacronym{orm}{ORM}{Object-Relational Mapping}
\newacronym{ide}{IDE}{Integrated Development Environment}
\newacronym{sdk}{SDK}{Software Development Kit}
\newacronym{cli}{CLI}{Command Line Interface}
\newacronym{gui}{GUI}{Graphical User Interface}

% === Bases de datos ===
\newacronym{dbms}{DBMS}{Database Management System}
\newacronym{nosql}{NoSQL}{Not Only SQL}
\newacronym{crud}{CRUD}{Create, Read, Update, Delete}
\newacronym{acid}{ACID}{Atomicity, Consistency, Isolation, Durability}

% === Seguridad ===
\newacronym{ssl}{SSL}{Secure Sockets Layer}
\newacronym{tls}{TLS}{Transport Layer Security}
\newacronym{jwt}{JWT}{JSON Web Token}
\newacronym{oauth}{OAuth}{Open Authorization}
\newacronym{rbac}{RBAC}{Role-Based Access Control}

% === Redes ===
\newacronym{ip}{IP}{Internet Protocol}
\newacronym{tcp}{TCP}{Transmission Control Protocol}
\newacronym{udp}{UDP}{User Datagram Protocol}
\newacronym{dns}{DNS}{Domain Name System}
\newacronym{cdn}{CDN}{Content Delivery Network}
\newacronym{vpn}{VPN}{Virtual Private Network}

% === Inteligencia Artificial ===
\newacronym{ia}{IA}{Inteligencia Artificial}
\newacronym{ml}{ML}{Machine Learning}
\newacronym{dl}{DL}{Deep Learning}
\newacronym{nlp}{NLP}{Natural Language Processing}
\newacronym{cnn}{CNN}{Convolutional Neural Network}
\newacronym{rnn}{RNN}{Recurrent Neural Network}
\newacronym{lstm}{LSTM}{Long Short-Term Memory}
\newacronym{gan}{GAN}{Generative Adversarial Network}
\newacronym{llm}{LLM}{Large Language Model}

% === Cloud y DevOps ===
\newacronym{aws}{AWS}{Amazon Web Services}
\newacronym{gcp}{GCP}{Google Cloud Platform}
\newacronym{iaas}{IaaS}{Infrastructure as a Service}
\newacronym{paas}{PaaS}{Platform as a Service}
\newacronym{saas}{SaaS}{Software as a Service}
\newacronym{k8s}{K8s}{Kubernetes}

%%%%%%%%%%%%%%%%%%%%%%%%%%%%%%%%%%%%%%%%%%%%%%%%%%%%%%%%%%%%%%%%%%%%%%%%
% GLOSARIO DE TÉRMINOS
%
% Formato: \newglossaryentry{etiqueta}{name={término}, description={definición}}
% Uso en el texto: \gls{etiqueta} (singular), \glspl{etiqueta} (plural)
%%%%%%%%%%%%%%%%%%%%%%%%%%%%%%%%%%%%%%%%%%%%%%%%%%%%%%%%%%%%%%%%%%%%%%%%

% === Conceptos de programación ===
\newglossaryentry{algoritmo}{
  name={algoritmo},
  description={Conjunto ordenado y finito de operaciones que permite hallar la solución de un problema. Los algoritmos son la base de la programación y definen los pasos lógicos para resolver tareas computacionales}
}

\newglossaryentry{framework}{
  name={framework},
  plural={frameworks},
  description={Estructura conceptual y tecnológica de soporte definido, normalmente con artefactos o módulos de software concretos, que puede servir de base para la organización y desarrollo de software}
}

\newglossaryentry{backend}{
  name={backend},
  description={Parte del software que procesa la entrada desde el frontend, gestiona la lógica de negocio y se comunica con la base de datos. También conocido como lado del servidor}
}

\newglossaryentry{frontend}{
  name={frontend},
  description={Parte del software que interactúa directamente con el usuario. Incluye la interfaz gráfica, formularios y todos los elementos visuales de una aplicación}
}

\newglossaryentry{middleware}{
  name={middleware},
  description={Software que actúa como puente entre el sistema operativo o base de datos y las aplicaciones, especialmente en una red. Facilita la comunicación y gestión de datos entre sistemas distribuidos}
}

\newglossaryentry{compilador}{
  name={compilador},
  description={Programa informático que traduce código escrito en un lenguaje de programación (código fuente) a otro lenguaje (código objeto), típicamente código máquina ejecutable}
}

\newglossaryentry{interprete}{
  name={intérprete},
  description={Programa que ejecuta instrucciones escritas en un lenguaje de programación línea por línea, sin necesidad de compilación previa. Python y JavaScript son ejemplos de lenguajes interpretados}
}

% === Arquitectura de software ===
\newglossaryentry{microservicio}{
  name={microservicio},
  plural={microservicios},
  description={Estilo arquitectónico que estructura una aplicación como una colección de servicios pequeños, autónomos y débilmente acoplados. Cada microservicio implementa una funcionalidad de negocio específica}
}

\newglossaryentry{monolito}{
  name={monolito},
  description={Arquitectura de software donde todos los componentes de la aplicación están interconectados y son interdependientes, desplegándose como una única unidad}
}

\newglossaryentry{escalabilidad}{
  name={escalabilidad},
  description={Capacidad de un sistema para manejar una cantidad creciente de trabajo, o su potencial para ser ampliado para acomodar ese crecimiento. Puede ser vertical (más recursos) u horizontal (más instancias)}
}

\newglossaryentry{latencia}{
  name={latencia},
  description={Tiempo que transcurre desde que se envía una solicitud hasta que se recibe la respuesta. En sistemas distribuidos, se mide típicamente en milisegundos}
}

\newglossaryentry{throughput}{
  name={throughput},
  description={Cantidad de datos o transacciones que un sistema puede procesar en una unidad de tiempo. Es una medida clave del rendimiento de sistemas}
}

% === Bases de datos ===
\newglossaryentry{transaccion}{
  name={transacción},
  plural={transacciones},
  description={Unidad lógica de trabajo en una base de datos que debe completarse en su totalidad o no ejecutarse en absoluto. Las transacciones garantizan la integridad de los datos}
}

\newglossaryentry{indice}{
  name={índice},
  sort={indice},
  plural={índices},
  description={Estructura de datos que mejora la velocidad de las operaciones de búsqueda en una tabla de base de datos, a costa de espacio adicional y tiempo de escritura}
}

\newglossaryentry{normalizacion}{
  name={normalización},
  description={Proceso de organización de datos en una base de datos relacional para reducir la redundancia y mejorar la integridad de los datos. Incluye varias formas normales (1NF, 2NF, 3NF, BCNF)}
}

\newglossaryentry{query}{
  name={query},
  plural={queries},
  description={Consulta o solicitud de datos a una base de datos. En SQL, las queries se escriben usando comandos como SELECT, INSERT, UPDATE y DELETE}
}

% === Seguridad informática ===
\newglossaryentry{hash}{
  name={hash},
  plural={hashes},
  description={Función que convierte una entrada de datos de cualquier tamaño en una salida de tamaño fijo. Se usa para verificar integridad de datos y almacenar contraseñas de forma segura}
}

\newglossaryentry{cifrado}{
  name={cifrado},
  description={Proceso de codificación de información para que solo las partes autorizadas puedan acceder a ella. Puede ser simétrico (misma clave) o asimétrico (par de claves pública/privada)}
}

\newglossaryentry{firewall}{
  name={firewall},
  description={Sistema de seguridad de red que monitoriza y controla el tráfico de red entrante y saliente según reglas de seguridad predeterminadas}
}

\newglossaryentry{token}{
  name={token},
  plural={tokens},
  description={Cadena de caracteres que representa una credencial de seguridad o sesión. En autenticación, los tokens permiten verificar la identidad sin transmitir contraseñas}
}

% === Metodologías y procesos ===
\newglossaryentry{sprint}{
  name={sprint},
  plural={sprints},
  description={Período de tiempo fijo (generalmente 2-4 semanas) durante el cual se completa un conjunto específico de trabajo en metodologías ágiles como Scrum}
}

\newglossaryentry{refactorizacion}{
  name={refactorización},
  description={Proceso de reestructurar código existente sin cambiar su comportamiento externo. Mejora la legibilidad, reduce la complejidad y facilita el mantenimiento}
}

\newglossaryentry{deploy}{
  name={deploy},
  plural={deploys},
  description={Proceso de poner una aplicación o actualización a disposición de los usuarios finales. Incluye la instalación, configuración y activación del software en el entorno de producción}
}

\newglossaryentry{pipeline}{
  name={pipeline},
  description={Secuencia automatizada de pasos para compilar, probar y desplegar software. En CI/CD, los pipelines automatizan el flujo desde el código fuente hasta producción}
}

% === Conceptos de LaTeX ===
\newglossaryentry{preambulo}{
  name={preámbulo},
  description={Parte del documento \LaTeX{} entre \texttt{\textbackslash documentclass} y \texttt{\textbackslash begin\{document\}}. Contiene la configuración del documento y la carga de paquetes}
}

\newglossaryentry{macro}{
  name={macro},
  plural={macros},
  description={Comando definido por el usuario en \LaTeX{} que representa una secuencia de instrucciones. Permite automatizar tareas repetitivas y crear abstracciones}
}

\newglossaryentry{flotante}{
  name={flotante},
  plural={flotantes},
  description={Elemento (figura o tabla) que \LaTeX{} puede mover de su posición en el código fuente para optimizar la maquetación. Se controla con especificadores como [htbp]}
}

\newglossaryentry{entorno}{
  name={entorno},
  plural={entornos},
  description={Bloque de código en \LaTeX{} delimitado por \texttt{\textbackslash begin\{nombre\}} y \texttt{\textbackslash end\{nombre\}}. Define un contexto especial para el contenido}
}
