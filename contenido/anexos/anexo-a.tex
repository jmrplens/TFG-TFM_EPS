\chapter{Manual de usuario}
\label{anexo:manual}

Este anexo contiene el manual de usuario del sistema desarrollado.

\section{Requisitos del sistema}

Para utilizar el sistema es necesario disponer de:

\begin{itemize}
  \item Navegador web moderno (Chrome 90+, Firefox 88+, Safari 14+, Edge 90+)
  \item Conexión a Internet estable
  \item Resolución de pantalla mínima de 1280x720 píxeles
\end{itemize}

\section{Acceso al sistema}

\subsection{Inicio de sesión}

Para acceder al sistema:

\begin{enumerate}
  \item Abra su navegador web
  \item Navegue a la URL del sistema: \url{https://ejemplo.ua.es}
  \item Introduzca sus credenciales (usuario y contraseña)
  \item Pulse el botón "Iniciar sesión"
\end{enumerate}

\subsection{Recuperación de contraseña}

Si ha olvidado su contraseña:

\begin{enumerate}
  \item Pulse en "¿Olvidó su contraseña?"
  \item Introduzca su correo electrónico
  \item Recibirá un enlace para restablecer la contraseña
  \item Siga las instrucciones del correo
\end{enumerate}

\section{Interfaz principal}

La interfaz principal se divide en las siguientes áreas:

\begin{description}
  \item[Barra de navegación:] Situada en la parte superior, permite acceder a las diferentes secciones.
  \item[Panel lateral:] Muestra el menú de opciones según el rol del usuario.
  \item[Área de contenido:] Zona principal donde se muestra la información.
  \item[Pie de página:] Contiene información de contacto y enlaces útiles.
\end{description}

\section{Funcionalidades principales}

\subsection{Gestión de datos}

Para gestionar datos en el sistema:

\begin{enumerate}
  \item Acceda a la sección "Datos" desde el menú lateral
  \item Seleccione la opción deseada:
    \begin{itemize}
      \item "Nuevo" para crear un registro
      \item "Editar" para modificar un registro existente
      \item "Eliminar" para borrar un registro
    \end{itemize}
  \item Complete el formulario correspondiente
  \item Pulse "Guardar" para confirmar los cambios
\end{enumerate}

\subsection{Generación de informes}

Para generar un informe:

\begin{enumerate}
  \item Acceda a la sección "Informes"
  \item Seleccione el tipo de informe
  \item Configure los filtros y parámetros
  \item Pulse "Generar"
  \item El informe se descargará en formato PDF
\end{enumerate}

\section{Preguntas frecuentes}

\begin{description}
  \item[¿Cómo cambio mi contraseña?] Acceda a su perfil y seleccione "Cambiar contraseña".
  
  \item[¿Puedo exportar mis datos?] Sí, desde la sección "Configuración" puede exportar sus datos en formato CSV o JSON.
  
  \item[¿Cómo contacto con soporte?] Envíe un correo a soporte@ejemplo.ua.es o utilice el formulario de contacto.
\end{description}
