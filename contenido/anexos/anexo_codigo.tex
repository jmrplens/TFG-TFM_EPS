%!TEX root = ../../main.tex

\chapter{Manual de Estilos de Código}\label{anexo:codigo}

Este anexo documenta el uso de los entornos de código disponibles en la plantilla, basados en los estilos de \textbf{Visual Studio Code}.

\section{Temas Disponibles}

La plantilla ofrece tres temas principales:

\begin{table}[H]
\centering
\caption{Temas de código disponibles}
\label{tab:temas-codigo}
\begin{tabular}{llp{5cm}}
\toprule
\textbf{Tema} & \textbf{Entorno} & \textbf{Uso recomendado} \\
\midrule
VS Code Light & \texttt{pythoncode}, etc. & Impresión, documentos formales \\
VS Code Dark & \texttt{pythoncodeDark}, etc. & Presentaciones, lectura en pantalla \\
Simple & \texttt{codigosimple} & Estilo minimalista con borde negro \\
\bottomrule
\end{tabular}
\end{table}

\subsection{Ejemplo del tema Simple}

El tema Simple proporciona un estilo minimalista con borde negro, sin colores de fondo:

\begin{codigosimple}{python}{Ejemplo de código simple}
def factorial(n):
    if n <= 1:
        return 1
    return n * factorial(n - 1)
\end{codigosimple}

\section{Nomenclatura de Entornos}

Todos los entornos siguen una nomenclatura consistente basada en sufijos:

\begin{table}[H]
\centering
\caption{Sistema de sufijos para entornos de código}
\label{tab:sufijos-codigo}
\begin{tabular}{llp{6cm}}
\toprule
\textbf{Sufijo} & \textbf{Ejemplo} & \textbf{Descripción} \\
\midrule
(ninguno) & \texttt{pythoncode} & Tema Light con números de línea \\
\texttt{NN} & \texttt{pythoncodeNN} & Tema Light sin números de línea \\
\texttt{Dark} & \texttt{pythoncodeDark} & Tema Dark con números de línea \\
\texttt{DarkNN} & \texttt{pythoncodeDarkNN} & Tema Dark sin números de línea \\
\bottomrule
\end{tabular}
\end{table}

\section{Lenguajes Predefinidos}

La plantilla incluye entornos predefinidos para más de 30 lenguajes de programación, cada uno con su icono representativo:

\begin{table}[H]
\centering
\caption{Lenguajes de programación con entornos predefinidos}
\label{tab:lenguajes-codigo}
\begin{tabular}{llll}
\toprule
\textbf{Lenguaje} & \textbf{Entorno} & \textbf{Icono} & \textbf{Pygments} \\
\midrule
Python & \texttt{pythoncode} & \faIcon{python} & python \\
JavaScript & \texttt{jscode} & \faIcon{js} & javascript \\
TypeScript & \texttt{tscode} & \faIcon{js} & typescript \\
Java & \texttt{javacode} & \faIcon{java} & java \\
C & \texttt{ccode} & \faIcon{copyright} & c \\
C++ & \texttt{cppcode} & \faIcon{copyright} & cpp \\
C\# & \texttt{csharpcode} & \faIcon{windows} & csharp \\
Go & \texttt{gocode} & \faIcon{google} & go \\
Rust & \texttt{rustcode} & \faIcon{rust} & rust \\
PHP & \texttt{phpcode} & \faIcon{php} & php \\
Ruby & \texttt{rubycode} & \faIcon{gem} & ruby \\
R & \texttt{rcode} & \faIcon{r-project} & r \\
Swift & \texttt{swiftcode} & \faIcon{apple} & swift \\
Kotlin & \texttt{kotlincode} & \faIcon{android} & kotlin \\
\bottomrule
\end{tabular}
\end{table}

\begin{table}[H]
\centering
\caption{Lenguajes web y de datos}
\label{tab:lenguajes-web}
\begin{tabular}{llll}
\toprule
\textbf{Lenguaje} & \textbf{Entorno} & \textbf{Icono} & \textbf{Pygments} \\
\midrule
HTML & \texttt{htmlcode} & \faIcon{html5} & html \\
CSS & \texttt{csscode} & \faIcon{css3-alt} & css \\
SASS/SCSS & \texttt{sasscode} & \faIcon{sass} & sass \\
JSON & \texttt{jsoncode} & \faIcon{database} & json \\
XML & \texttt{xmlcode} & \faIcon{file-code} & xml \\
YAML & \texttt{yamlcode} & \faIcon{file-alt} & yaml \\
Markdown & \texttt{markdowncode} & \faIcon{markdown} & markdown \\
SQL & \texttt{sqlcode} & \faIcon{database} & sql \\
\bottomrule
\end{tabular}
\end{table}

\begin{table}[H]
\centering
\caption{Lenguajes de sistemas y DevOps}
\label{tab:lenguajes-sistemas}
\begin{tabular}{llll}
\toprule
\textbf{Lenguaje} & \textbf{Entorno} & \textbf{Icono} & \textbf{Pygments} \\
\midrule
Bash & \texttt{bashcode} & \faIcon{terminal} & bash \\
PowerShell & \texttt{powershellcode} & \faIcon{terminal} & powershell \\
Docker & \texttt{dockercode} & \faIcon{docker} & docker \\
Makefile & \texttt{makefilecode} & \faIcon{cogs} & makefile \\
Git & \texttt{gitcode} & \faIcon{git-alt} & text \\
\bottomrule
\end{tabular}
\end{table}

\section{Ejemplos de Uso}

\subsection{Tema VS Code Light}

\subsubsection{Con números de línea}

\begin{pythoncode}[title={Ejemplo: Función recursiva}]
def factorial(n):
    """Calcula el factorial de n."""
    if n <= 1:
        return 1
    return n * factorial(n - 1)

\# Ejemplo de uso
resultado = factorial(5)
print(f"5! = {resultado}")
\end{pythoncode}

\noindent Uso en \LaTeX{}:

\begin{latexcode}
\begin{pythoncode}[title={Ejemplo: Función recursiva}]
def factorial(n):
    if n <= 1:
        return 1
    return n * factorial(n - 1)
\end{pythoncode}
\end{latexcode}

\subsubsection{Sin números de línea}

\begin{pythoncodeNN}
print("Código sin números de línea")
x = 10 + 20
\end{pythoncodeNN}

\noindent Uso en \LaTeX{}:

\begin{latexcode}
\begin{pythoncodeNN}
print("Código sin números de línea")
\end{pythoncodeNN}
\end{latexcode}

\subsection{Tema VS Code Dark}

\subsubsection{Con números de línea}

\begin{pythoncodeDark}[title={Tema oscuro con números}]
import numpy as np
import matplotlib.pyplot as plt

x = np.linspace(0, 2*np.pi, 100)
y = np.sin(x)
plt.plot(x, y)
\end{pythoncodeDark}

\subsubsection{Sin números de línea}

\begin{pythoncodeDarkNN}
\# Tema oscuro sin números
mensaje = "Visual Studio Code Dark"
print(mensaje)
\end{pythoncodeDarkNN}

\section{Entorno Genérico}

Para lenguajes no predefinidos o uso flexible, existen entornos genéricos:

\begin{table}[H]
\centering
\caption{Entornos genéricos de código}
\label{tab:entornos-genericos}
\begin{tabular}{llp{5cm}}
\toprule
\textbf{Entorno} & \textbf{Sintaxis} & \textbf{Descripción} \\
\midrule
\texttt{codigo} & \verb|\begin{codigo}{lang}| & Light + números \\
\texttt{codigoNN} & \verb|\begin{codigoNN}{lang}| & Light sin números \\
\texttt{codigoDark} & \verb|\begin{codigoDark}{lang}| & Dark + números \\
\texttt{codigoDarkNN} & \verb|\begin{codigoDarkNN}{lang}| & Dark sin números \\
\bottomrule
\end{tabular}
\end{table}

\noindent Ejemplo con Haskell:

\begin{codigo}{haskell}
-- Función quicksort en Haskell
quicksort :: Ord a => [a] -> [a]
quicksort [] = []
quicksort (x:xs) = quicksort menores ++ [x] ++ quicksort mayores
  where
    menores = [y | y <- xs, y <= x]
    mayores = [y | y <- xs, y > x]
\end{codigo}

\section{Consideraciones Especiales}

\subsection{Código C/C++ con includes}

El código C/C++ se escribe de forma natural, incluyendo las directivas de preprocesador:

\begin{cppcode}[title={Ejemplo C++ con includes}]
#include <iostream>
#include <vector>

int main() {
    std::vector<int> v = {1, 2, 3};
    for (int n : v) {
        std::cout << n << " ";
    }
    return 0;
}
\end{cppcode}

\subsection{Títulos personalizados}

Todos los entornos aceptan un parámetro opcional \texttt{title}:

\begin{bashcode}[title={Script de instalación}]
#!/bin/bash
echo "Instalando dependencias..."
apt-get update && apt-get install -y python3
\end{bashcode}

\subsection{Requisitos de compilación}

Para que los entornos de código funcionen correctamente:

\begin{enumerate}
    \item Compilar con \textbf{LuaLaTeX} o XeLaTeX
    \item Habilitar \texttt{-shell-escape}
    \item Tener instalado \textbf{latexminted} (\texttt{pip install latexminted})
\end{enumerate}

\section{Referencia Rápida}

\begin{table}[H]
\centering
\caption{Matriz de entornos por tema y numeración}
\label{tab:matriz-entornos}
\begin{tabular}{lcccc}
\toprule
\textbf{Lenguaje} & \textbf{Light+Num} & \textbf{Light-Num} & \textbf{Dark+Num} & \textbf{Dark-Num} \\
\midrule
Python & \texttt{pythoncode} & \texttt{pythoncodeNN} & \texttt{pythoncodeDark} & \texttt{pythoncodeDarkNN} \\
JavaScript & \texttt{jscode} & \texttt{jscodeNN} & \texttt{jscodeDark} & \texttt{jscodeDarkNN} \\
Java & \texttt{javacode} & \texttt{javacodeNN} & \texttt{javacodeDark} & \texttt{javacodeDarkNN} \\
C++ & \texttt{cppcode} & \texttt{cppcodeNN} & \texttt{cppcodeDark} & \texttt{cppcodeDarkNN} \\
HTML & \texttt{htmlcode} & \texttt{htmlcodeNN} & \texttt{htmlcodeDark} & \texttt{htmlcodeDarkNN} \\
SQL & \texttt{sqlcode} & \texttt{sqlcodeNN} & \texttt{sqlcodeDark} & \texttt{sqlcodeDarkNN} \\
Bash & \texttt{bashcode} & \texttt{bashcodeNN} & \texttt{bashcodeDark} & \texttt{bashcodeDarkNN} \\
Genérico & \texttt{codigo} & \texttt{codigoNN} & \texttt{codigoDark} & \texttt{codigoDarkNN} \\
\bottomrule
\end{tabular}
\end{table}
