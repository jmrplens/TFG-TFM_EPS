%% =============================================================================
%% ANEXO: TÉCNICAS AVANZADAS DE LATEX
%% =============================================================================
\chapter{Técnicas Avanzadas de \LaTeX}
\label{anexo:tecnicas-avanzadas}

Este anexo presenta técnicas avanzadas de \LaTeX{} que pueden ser útiles en la 
elaboración de un \gls{tfg} o \gls{tfm}: tablas rotadas, páginas en horizontal,
inclusión de documentos PDF externos, y otras funcionalidades.

%% =============================================================================
\section{Tablas Rotadas (Sideways Tables)}
\label{sec:tablas-rotadas}

Cuando una tabla tiene muchas columnas y no cabe en el ancho de página normal,
se puede rotar \qty{90}{\degree} para aprovechar el alto de la página como ancho.
Para ello se usa el entorno \texttt{sidewaystable} del paquete \texttt{rotating}.

\subsection{Ejemplo de tabla rotada}

El siguiente código genera una tabla que ocupa toda la página en horizontal:

\begin{latexcode}[title={Código de tabla rotada}]
\begin{sidewaystable}
  \centering
  \caption{Comparativa de características por módulo del sistema}
  \label{tab:comparativa-modulos}
  \begin{tabular}{lcccccccccc}
    \toprule
    \textbf{Módulo} & \textbf{Líneas} & \textbf{Clases} & ... \\
    \midrule
    ...
    \bottomrule
  \end{tabular}
\end{sidewaystable}
\end{latexcode}

% La tabla real se muestra a continuación (rotada 90 grados)
\begin{sidewaystable}
  \centering
  \caption{Comparativa completa de características por módulo del sistema desarrollado}
  \label{tab:comparativa-modulos}
  \begin{tabular}{lcccccccccccc}
    \toprule
    \textbf{Módulo} & 
    \textbf{Líneas} & 
    \textbf{Clases} & 
    \textbf{Funciones} & 
    \textbf{Tests} & 
    \textbf{Cobertura} & 
    \textbf{Complejidad} & 
    \textbf{Dependencias} &
    \textbf{Memoria} &
    \textbf{CPU} &
    \textbf{Latencia} &
    \textbf{Estado} \\
    \midrule
    Autenticación    & 2.450  & 12 & 45  & 89  & 94\% & Media & 5  & 128 MB & 2\%  & 45 ms  & \checkmark \\
    Base de datos    & 3.200  & 18 & 67  & 120 & 91\% & Alta  & 8  & 256 MB & 5\%  & 12 ms  & \checkmark \\
    API REST         & 4.100  & 25 & 98  & 156 & 88\% & Media & 12 & 192 MB & 8\%  & 23 ms  & \checkmark \\
    Interfaz web     & 5.600  & 45 & 134 & 89  & 72\% & Baja  & 18 & 384 MB & 12\% & 180 ms & En desarrollo \\
    Notificaciones   & 1.800  & 8  & 32  & 67  & 96\% & Baja  & 4  & 64 MB  & 1\%  & 8 ms   & \checkmark \\
    Caché            & 980    & 5  & 18  & 45  & 98\% & Baja  & 2  & 512 MB & 3\%  & 2 ms   & \checkmark \\
    Logs y métricas  & 1.200  & 6  & 24  & 38  & 85\% & Baja  & 3  & 96 MB  & 2\%  & 5 ms   & \checkmark \\
    Procesamiento    & 2.800  & 14 & 56  & 78  & 82\% & Alta  & 7  & 768 MB & 25\% & 350 ms & \checkmark \\
    Exportación      & 1.500  & 9  & 28  & 52  & 90\% & Media & 6  & 128 MB & 4\%  & 120 ms & \checkmark \\
    Configuración    & 650    & 4  & 15  & 34  & 100\%& Baja  & 1  & 32 MB  & 0\%  & 1 ms   & \checkmark \\
    \midrule
    \textbf{Total}   & \textbf{24.280} & \textbf{146} & \textbf{517} & \textbf{768} & \textbf{87\%} & -- & \textbf{66} & -- & -- & -- & -- \\
    \bottomrule
  \end{tabular}
  
  \smallskip
  \footnotesize
  \textbf{Notas:} Líneas = líneas de código fuente; Cobertura = porcentaje de código cubierto por tests;
  Complejidad = complejidad ciclomática promedio; Latencia = tiempo de respuesta medio.
\end{sidewaystable}

\subsection{Cuándo usar tablas rotadas}

Las tablas rotadas son útiles cuando:
\begin{itemize}
  \item La tabla tiene más de 8-10 columnas
  \item Los encabezados de columna son largos
  \item Se necesita mostrar datos comparativos extensos
  \item Una tabla horizontal no cabría sin reducir excesivamente el tamaño de fuente
\end{itemize}

%% =============================================================================
\section{Páginas en Horizontal (Landscape)}
\label{sec:paginas-landscape}

Para insertar una o varias páginas en orientación horizontal dentro de un documento
vertical, se usa el entorno \texttt{landscape} del paquete \texttt{pdflscape}. 
A diferencia de \texttt{lscape}, este paquete también rota la página en el visor PDF.

\subsection{Ejemplo de página landscape}

\begin{latexcode}[title={Código para página horizontal}]
\begin{landscape}
  \section{Diagrama de arquitectura}
  
  % Contenido que requiere más ancho
  \begin{figure}[H]
    \centering
    % Diagrama TikZ o imagen amplia
    \caption{Arquitectura del sistema}
  \end{figure}
\end{landscape}
\end{latexcode}

% Página landscape real
\begin{landscape}
\subsection{Diagrama de flujo del sistema (página horizontal)}

Esta página está en orientación horizontal para mostrar un diagrama amplio.

\begin{figure}[H]
  \centering
  \begin{tikzpicture}[
    node distance=1.5cm and 2cm,
    block/.style={rectangle, draw, fill=blue!10, text width=2.5cm, text centered, 
                  rounded corners, minimum height=1cm},
    decision/.style={diamond, draw, fill=yellow!20, text width=2cm, 
                     text centered, aspect=2},
    arrow/.style={thick,->,>=stealth}
  ]
    % Nodos
    \node[block] (inicio) {Inicio};
    \node[block, right=of inicio] (auth) {Autenticación};
    \node[decision, right=of auth] (valido) {¿Válido?};
    \node[block, right=of valido] (menu) {Menú principal};
    \node[block, above right=1cm and 2cm of menu] (modulo1) {Módulo A};
    \node[block, right=of menu] (modulo2) {Módulo B};
    \node[block, below right=1cm and 2cm of menu] (modulo3) {Módulo C};
    \node[block, right=3cm of modulo2] (proceso) {Procesamiento};
    \node[block, right=of proceso] (resultado) {Resultados};
    \node[decision, right=of resultado] (guardar) {¿Guardar?};
    \node[block, right=of guardar] (bd) {Base de datos};
    \node[block, below=of valido] (error) {Error};
    \node[block, below=of guardar] (fin) {Fin};
    
    % Flechas
    \draw[arrow] (inicio) -- (auth);
    \draw[arrow] (auth) -- (valido);
    \draw[arrow] (valido) -- node[above] {Sí} (menu);
    \draw[arrow] (valido) -- node[right] {No} (error);
    \draw[arrow] (error) -| (auth);
    \draw[arrow] (menu) -- (modulo1);
    \draw[arrow] (menu) -- (modulo2);
    \draw[arrow] (menu) -- (modulo3);
    \draw[arrow] (modulo1) -| (proceso);
    \draw[arrow] (modulo2) -- (proceso);
    \draw[arrow] (modulo3) -| (proceso);
    \draw[arrow] (proceso) -- (resultado);
    \draw[arrow] (resultado) -- (guardar);
    \draw[arrow] (guardar) -- node[above] {Sí} (bd);
    \draw[arrow] (guardar) -- node[right] {No} (fin);
    \draw[arrow] (bd) |- (fin);
  \end{tikzpicture}
  \caption{Diagrama de flujo completo del sistema en página horizontal}
  \label{fig:diagrama-landscape}
\end{figure}

\vspace{1cm}

\begin{table}[H]
  \centering
  \caption{Matriz de trazabilidad requisitos-módulos (aprovechando el ancho de página landscape)}
  \label{tab:trazabilidad-landscape}
  \begin{tabular}{l|ccccccccccccc}
    \toprule
    & \textbf{RF01} & \textbf{RF02} & \textbf{RF03} & \textbf{RF04} & \textbf{RF05} & 
      \textbf{RF06} & \textbf{RF07} & \textbf{RF08} & \textbf{RF09} & \textbf{RF10} &
      \textbf{RF11} & \textbf{RF12} & \textbf{RF13} \\
    \midrule
    Módulo A & $\bullet$ & $\bullet$ &           & $\bullet$ &           &           & $\bullet$ &           &           & $\bullet$ &           &           & $\bullet$ \\
    Módulo B &           & $\bullet$ & $\bullet$ &           & $\bullet$ & $\bullet$ &           &           & $\bullet$ &           & $\bullet$ &           &           \\
    Módulo C & $\bullet$ &           &           & $\bullet$ &           &           &           & $\bullet$ &           &           &           & $\bullet$ &           \\
    Módulo D &           &           & $\bullet$ &           & $\bullet$ &           & $\bullet$ &           & $\bullet$ &           &           &           & $\bullet$ \\
    Módulo E &           & $\bullet$ &           &           &           & $\bullet$ &           & $\bullet$ &           & $\bullet$ & $\bullet$ &           &           \\
    \bottomrule
  \end{tabular}
\end{table}
\end{landscape}

\subsection{Cuándo usar páginas landscape}

Las páginas en horizontal son apropiadas para:
\begin{itemize}
  \item Diagramas de flujo o arquitectura complejos
  \item Cronogramas o diagramas de Gantt
  \item Matrices de trazabilidad
  \item Capturas de pantalla de aplicaciones
  \item Tablas muy anchas que no justifican rotación completa
\end{itemize}

%% =============================================================================
\section{Inclusión de Documentos PDF Externos}
\label{sec:inclusion-pdf}

El paquete \texttt{pdfpages} permite incluir páginas de documentos PDF externos
en el documento \LaTeX. Esto es útil para adjuntar:
\begin{itemize}
  \item Artículos o papers de referencia
  \item Documentación técnica de terceros
  \item Certificados o autorizaciones
  \item Manuales de usuario existentes
  \item Hojas de datos (datasheets)
\end{itemize}

\subsection{Sintaxis básica}

\begin{latexcode}[title={Inclusión de PDF externo}]
% Incluir todas las páginas
\includepdf[pages=-]{ruta/documento.pdf}

% Incluir páginas específicas
\includepdf[pages={1,3,5-8}]{documento.pdf}

% Incluir con opciones
\includepdf[
  pages=-,
  scale=0.9,
  pagecommand={\thispagestyle{plain}}
]{documento.pdf}
\end{latexcode}

\subsection{Opciones más utilizadas}

\begin{table}[H]
  \centering
  \caption{Opciones principales de \texttt{\textbackslash includepdf}}
  \label{tab:opciones-includepdf}
  \begin{tabular}{lp{8cm}}
    \toprule
    \textbf{Opción} & \textbf{Descripción} \\
    \midrule
    \texttt{pages=-} & Incluye todas las páginas \\
    \texttt{pages=\{1,3,5-8\}} & Incluye páginas específicas \\
    \texttt{scale=0.9} & Escala el documento (0.9 = 90\%) \\
    \texttt{landscape} & Rota las páginas 90 grados \\
    \texttt{nup=2x2} & Coloca varias páginas en una (2 filas × 2 columnas) \\
    \texttt{frame} & Añade un marco alrededor de cada página \\
    \texttt{pagecommand=\{\}} & Comando a ejecutar en cada página incluida \\
    \texttt{addtotoc} & Añade entrada al índice de contenidos \\
    \bottomrule
  \end{tabular}
\end{table}

\subsection{Ejemplo: Documento PDF incluido}

A continuación se incluye un documento PDF de ejemplo que ha sido generado 
independientemente. Se incluyen sus dos páginas con un marco y una escala 
ligeramente reducida:

% Incluir el PDF externo de ejemplo
\includepdf[
  pages=-,
  scale=0.85,
  frame,
  pagecommand={\thispagestyle{plain}},
]{recursos/pdf_externos/documento_ejemplo.pdf}

\subsection{Múltiples páginas en una hoja}

Para ahorrar espacio, se pueden incluir varias páginas del PDF en una sola 
hoja del documento:

\begin{latexcode}[title={Varias páginas PDF en una hoja}]
\includepdf[
  pages=-,
  nup=1x2,           % 1 columna, 2 filas
  landscape,         % Orientación horizontal
  frame,             % Marco visible
]{documento.pdf}
\end{latexcode}

%% =============================================================================
\section{Figuras de Ancho Completo}
\label{sec:figuras-ancho-completo}

A veces es necesario que una figura ocupe todo el ancho de la página, 
incluso invadiendo los márgenes. Para ello se combina el entorno 
\texttt{figure*} con ajustes de geometría.

\subsection{Figura que invade márgenes}

\begin{latexcode}[title={Figura de ancho completo}]
\begin{figure}[H]
  \centering
  \makebox[\textwidth][c]{%
    \includegraphics[width=1.2\textwidth]{imagen_ancha}
  }
  \caption{Imagen que ocupa más que el ancho del texto}
\end{figure}
\end{latexcode}

%% =============================================================================
\section{Notas al Margen}
\label{sec:notas-margen}

Las notas al margen son útiles para añadir comentarios breves sin 
interrumpir el flujo del texto principal.

\subsection{Uso básico}

\begin{latexcode}[title={Notas al margen}]
Texto principal del párrafo.\marginpar{Nota breve al margen}
\end{latexcode}

Este es un ejemplo de texto con una nota al margen.\marginpar{\footnotesize\textit{Esta es una nota al margen con información adicional.}}

Las notas al margen se colocan automáticamente en el lado exterior de la 
página (derecho en páginas impares, izquierdo en pares) cuando se usa 
impresión a doble cara.

%% =============================================================================
\section{Marcas de Agua}
\label{sec:marcas-agua}

Para documentos en borrador o confidenciales, se pueden añadir marcas de agua
usando el paquete \texttt{draftwatermark} o \texttt{background}.

\begin{latexcode}[title={Marca de agua con draftwatermark}]
% En el preámbulo:
\usepackage{draftwatermark}
\SetWatermarkText{BORRADOR}
\SetWatermarkScale{1.5}
\SetWatermarkColor[gray]{0.9}
\end{latexcode}

%% =============================================================================
\section{Texto en Columnas}
\label{sec:columnas}

Para secciones específicas que requieran formato en múltiples columnas
(como glosarios o listas de referencias), se puede usar el entorno
\texttt{multicols}:

\begin{latexcode}[title={Texto en dos columnas}]
\begin{multicols}{2}
  Contenido distribuido en dos columnas...
\end{multicols}
\end{latexcode}

\subsection{Ejemplo de texto en columnas}

\begin{multicols}{2}
\small
\textbf{Términos de red:}
\begin{itemize}[noitemsep]
  \item Router
  \item Switch
  \item Firewall
  \item Gateway
  \item DNS Server
  \item DHCP Server
\end{itemize}

\columnbreak

\textbf{Protocolos:}
\begin{itemize}[noitemsep]
  \item TCP/IP
  \item HTTP/HTTPS
  \item FTP/SFTP
  \item SSH
  \item SMTP
  \item DNS
\end{itemize}
\end{multicols}

%% =============================================================================
\section{Minipáginas y Cajas}
\label{sec:minipaginas}

Las minipáginas permiten crear bloques de contenido lado a lado:

\begin{latexcode}[title={Dos minipáginas lado a lado}]
\begin{minipage}[t]{0.45\textwidth}
  Contenido izquierdo...
\end{minipage}
\hfill
\begin{minipage}[t]{0.45\textwidth}
  Contenido derecho...
\end{minipage}
\end{latexcode}

\subsection{Ejemplo de minipáginas}

\noindent
\begin{minipage}[t]{0.48\textwidth}
  \textbf{Ventajas del sistema:}
  \begin{itemize}[noitemsep]
    \item Alta disponibilidad
    \item Escalabilidad horizontal
    \item Bajo coste de mantenimiento
    \item Interfaz intuitiva
    \item Documentación completa
  \end{itemize}
\end{minipage}
\hfill
\begin{minipage}[t]{0.48\textwidth}
  \textbf{Limitaciones conocidas:}
  \begin{itemize}[noitemsep]
    \item Requiere conexión a Internet
    \item No compatible con IE11
    \item Máximo 1000 usuarios simultáneos
    \item Sin soporte para móviles legacy
    \item Idiomas: solo ES/EN
  \end{itemize}
\end{minipage}

%% =============================================================================
\section{Resumen de Paquetes Utilizados}
\label{sec:resumen-paquetes}

\begin{table}[H]
  \centering
  \caption{Paquetes \LaTeX{} para técnicas avanzadas}
  \label{tab:paquetes-avanzados}
  \begin{tabular}{llp{6cm}}
    \toprule
    \textbf{Paquete} & \textbf{Uso} & \textbf{Comando principal} \\
    \midrule
    \texttt{rotating} & Tablas rotadas & \verb|\begin{sidewaystable}| \\
    \texttt{pdflscape} & Páginas landscape & \verb|\begin{landscape}| \\
    \texttt{pdfpages} & Incluir PDFs & \verb|\includepdf[options]{file}| \\
    \texttt{multicol} & Múltiples columnas & \verb|\begin{multicols}{n}| \\
    \texttt{geometry} & Márgenes personalizados & \verb|\newgeometry{...}| \\
    \texttt{fancyhdr} & Encabezados/pies & \verb|\pagestyle{fancy}| \\
    \texttt{draftwatermark} & Marcas de agua & \verb|\SetWatermarkText{...}| \\
    \bottomrule
  \end{tabular}
\end{table}
